\documentclass[utf8, diplomski, english, numeric]{fer}
\usepackage{booktabs}
\usepackage{enumitem}
\usepackage{algorithm}
\usepackage[noend]{algpseudocode}
\algnewcommand\And{\textbf{and}}
\algnewcommand\To{\textbf{to}}
\usepackage{verbatim}
\usepackage{float}
\usepackage[table,xcdraw]{xcolor}
\usepackage[final]{pdfpages}

\usepackage{listings}
\lstset{
    language=C,
    basicstyle=\fontsize{11}{13}\ttfamily,
    % numbers=left,
    frame=single,
    % numberstyle=\tiny,
    showstringspaces=false,
    tabsize=4
}

\usepackage{color}
\definecolor{bluekeywords}{rgb}{0.13,0.13,1}
\definecolor{greencomments}{rgb}{0,0.5,0}
\definecolor{redstrings}{rgb}{0.9,0,0}

\newtheorem{mydef}{Definition}

\begin{document}

% TODO: Navedite broj rada.
\thesisnumber{1982}

% TODO: Navedite naslov rada.
\title{Application of Genetic Programming for Real-Time Task Scheduling}

% TODO: Navedite svoje ime i prezime.
\author{Karla Salamun}

\maketitle

% Ispis stranice s napomenom o umetanju izvornika rada. Uklonite naredbu \izvornik ako želite izbaciti tu stranicu.
\includepdf{hr_0036492033_48.pdf}

% Dodavanje zahvale ili prazne stranice. Ako ne želite dodati zahvalu, naredbu ostavite radi prazne stranice.
\zahvala{}

\tableofcontents
\listoffigures
% \lstlistoflistings

\chapter*{Introduction}
Real-time systems have a wide range of applications, including monitoring systems in industrial plants, automotive, space missions, multimedia systems and consumer electronics.
Reliability of these systems depends on the timeliness and predictability of task execution.
The scope of this thesis considers firm real-time systems, where the completion of a tardy task brings no value to the system.
A typical example of such system is a multimedia system.
Although a certain degree of deadline miss ratio is allowed, the system will eventually suffer from a failure if consecutive instances of a task fail to complete before their deadlines.

There is a handful of formal algorithms for scheduling periodic firm real-time tasks proposed in literature.
The general idea of these algorithms is to resolve system overload by skipping some task instances in order to achieve the feasible load.
Making optimal use of skips has been proven to be an NP-hard problem \cite{queudet2012quality}.

The idea of this thesis is to investigate the possibility of using machine learning to optimize a heuristic for scheduling skippable periodic tasks.
More specifically, the goal is to build a scheduling heuristic such that the number of successfully completed task instances between skipped instances is maximized.
The scheduling algorithm is structured in two components: a meta-algorithm and a priority function.
A meta-algorithm is a method of assigning task instances (jobs) to resources with respect to task properties and system constraints.
It is defined for a given scheduling environment.
In this thesis, the considered environment is a single-machine environment where $n$ periodic tasks are processed on a single resource. 
The meta-algorithm encapsulates the priority function which defines priority values for the tasks.
This thesis deals with generating the priority function using genetic programming.

The thesis is organized as follows.
Chapter \ref{genprog} gives an introduction to optimization techniques, evolutionary algorithms and genetic programming.
The main concepts of multi-objective optimization and cooperative coevolution are described.
Chapter \ref{fm} describes the formal methods for handling transient and permanent overload conditions in real-time systems with emphasis on job skipping techniques.
Chapter \ref{implementation} presents detailed description of the implementation of a framework for genetic programming in C++ language.
The interfaces for applying evolved heuristic to task scheduling problem and evaluating its performace are also described.
In Chapter \ref{results}, the genetic programming scheduling approach is compared to formal scheduling algorithms. Moreover, the implemented framework for genetic programming is compared to an existing framework for evolutionary computation.
Additionally, the performance of the evolved heuristic implemented in FreeRTOS scheduler is presented. 
\chapter{Genetic Programming}
\section{Genetic Algorithms}
\section{Tree-Based Genetic Programming}
\section{Applications of Genetic Programming}
\chapter{Formal Methods of Scheduling Overloaded Real-Time Systems}
%%%% TODO ovdje pisati o podjeli zadataka na firm, soft, hard???

\section{Transient Overload}
\subsection{Aperiodic Overloads}
\subsection{Task Overruns}

%%%%%%%%%%%%%%%%%%%%%%%%%%%%%%%%%%%%%%%%%%%%%%%%%%%%%%%%%%%%%%%%%%%%%%%%%%%%%%%%%%%%%%%%%%%%%%
\section{Permanent Overload}

Permanent overload in periodic task systems occurs when the total utilization factor of the periodic task set is greater than one \cite{buttazzo2011hard}. 
This condition occurs either because of wrong estimation of the task execution time, unexpected activation of new periodic tasks, or the increase of activation rate of current tasks.
Methods for reducing permanent overload are the following:
\begin{itemize}
	\item{job skipping,}
	\item{period adaptation,}
	\item{service adaptation.}
\end{itemize}

\subsection{Job Skipping}
Job skipping method reduces system load by skipping some jobs in the task set. Jobs to be skipped are assigned in a way that the remainig jobs can be scheduled within their deadlines.
This method is applicable for firm tasks, as they allow a certain miss ratio. 
The quantified ratio of tasks that may not be executed is directly related to Quality of Service 
(QoS) metric. 
The algorithms presented in this section address the task miss ratio as a QoS concern, with the purpose of maximizing the QoS of periodic tasks.

A task model suitable for job skipping method is known in literature as the firm periodic model.
It was first described by Koren and Shasha \cite{koren1995skip}.
According to this model, every task is described as following:
\begin{align*}
\tau_i(C_i, T_i, D_i, S_i)
\end{align*}
where $C_i$ is the worst-case computation time, $T_i$ is the task period, $D_i$ is the relative deadline (assumed equal to the period). $S_i$ is the skip parameter and it marks the minimum distance between two consecutive skips, \(2 \leq S_i \leq \infty\).

An example of schedule with task skipping is given in figure \ref{skipover}. 
Since the utilization factor equals 
\begin{equation*}
U_p = \sum_{i=1}^{n} \frac{C_i}{T_i} = 1.17,
\end{equation*}
the system is permanently overloaded. 
However, if task \(\tau_2\) is skipped every three instances, the overload is resolved.
\\
\begin{figure}[ht]
    \centering
    \includegraphics[width=0.9\textwidth]{images/skipover.pdf}
    \caption{An example of resolving permanent overload by skipping method.}
    \label{skipover}
\end{figure}

Corresponding to the stated task model, every job of a periodic task can be red or blue.
A red job must be completed within its deadline, whereas a blue job can be aborted at any time 
\cite{buttazzo2011hard}. 
With respect to the skip parameter $S_i$, each scheduling algorithm needs to fulfill two conditions:
\begin{itemize}
	\item if a blue job is skipped, then the next \(S_i - 1\) jobs must be red,
	\item if a blue job completes successfully, the next job is also blue.
\end{itemize}
The authors proved that the problem of determining whether a set of periodic skippable tasks is schedulable is NP-hard \cite{koren1995skip}.

% Although the algorithms are not optimal, they become optimal under a certain condition, called the 
% \textit{deeply-red} condition. 
% The \textit{deeply-red} condition is the worst case condition for periodic task sets:
% all tasks are synchronously activated and the first \(S_i - 1\) instances of every task $\tau_i$ are red. 

For the given set of skippable periodic tasks \( \Gamma = {\tau_i(T_i,C_i,S_i)} \), the necessary schedulability condition is the following:
\begin{equation}
\sum_{i=1}^{n} \frac{C_i(S_i-1)}{T_i S_i} \leq 1.
\end{equation}

A sufficient condition for guaranteeing schedulability of a set of skippable tasks can be stated 
using the equivalent utilization factor, which is formally described in the following definition:
\newtheorem{mydef}{Definition}
\begin{mydef}
Given a set \( \Gamma = {\tau_i(T_i,C_i,S_i)} \) of n skippable periodic tasks, the equivalent utilization factor is defined as:
\begin{equation*}
U_p^\ast = \max_{L \geq 0} \frac{\sum_{i}D(i, [0,L])}{L}
\end{equation*}
where
\begin{equation*}
D(i, [0,L]) = \left\lfloor \frac{L}{T_i} - \frac{L}{T_i \cdot S_i}\right\rfloor \cdot C_i.
\end{equation*}
\end{mydef}

According to the sufficient schedulability condition, a set of skippable periodic tasks is schedulable if \( U_p^\ast \leq 1 \).

\subsubsection{RTO Algorithm}
\subsubsection{BWP Algorithm}
\subsubsection{RLP Algorithm}


\subsection{Period Adaptation}
\subsection{Service Adaptation}
\chapter{Implementation Overview}
\section{Generating Test Sets}
\section{Heuristics Evolution}
\subsection{Genotype Description}
\subsection{Operators}
\section{Heuristics Evaluation}
\subsection{Static Environment}
\subsection{Periodic Tasks}
\section{FreeRTOS Scheduler Modification}
\chapter{Results}
\section{Comparison of Formal Methods and Genetic Programming}
\section{Comparison of Implemented Framework and ECF Framework}
\section{Testing the FreeRTOS Modification}
Modification of the FreeRTOS kernel was tested on a FreeRTOS simulator for Linux systems which uses POSIX threads.
The test application creates multiple tasks with diverse period, duration and weight parameters. 
A code example for a periodic task is given in the following listing.
CPU utilization is simulated by the \verb$count$ variable whose inital value corresponds to the task processing time. 
The variable is decremented at every system tick and when it reaches zero, the \verb$vTaskDealyUntil()$ function is called. 
\begin{lstlisting}[frame=none, label={switchcontext}, caption={Task function simulating a periodic task.}, captionpos=b]
void TSK_A( void *pvParameters ) {
    TickType_t xLastWakeTimeA;
    const TickType_t xFrequency = PERIOD_A;
    TickType_t count = DURATION_A;

    TickType_t xNextTime;
    TickType_t xTime;
    xLastWakeTimeA = 0;
    
    for(;;) {
        xTime= xTaskGetTickCount();
        //While loop that simulates task duration
        while(count != 0) {
            if((xNextTime = xTaskGetTickCount()) > xTime) {
                count--;
                xTime = xNextTime;
            }
        }
        count = DURATION_A;
        vTaskDelayUntil(&xLastWakeTimeA, xFrequency);
    }
}
\end{lstlisting}
Events of interest - tick increments and context switches are monitored by using trace macros. 
The following trace macros were implemented:
\begin{itemize}
	\item \verb$traceTASK_INCREMENT_TICK( xTickCount )$,
	\item \verb$traceTASK_SWITCHED_IN()$,
	\item \verb$traceTASK_SWITCHED_OUT()$.
\end{itemize}
The \verb$traceTASK_INCREMENT_TICK()$ macro is called during the tick interrupt.
During a context switch, the \verb$traceTASK_SWITCHED_IN()$ macro contains the handle of the task about to enter the running state, while the \verb$traceTASK_SWITCHED_OUT()$ contains the handle of the task about to leave the running state \cite{freertosref}. 

For tracing the total tardiness in overload conditions, a global variable \verb$xTardiness$ is used.
The tardiness value is updated every time a task execution is finished.
\begin{lstlisting}[frame=none, label={tardiness}, caption={Updating the \texttt{xTardiness} variable.}, captionpos=b]
#if( configTRACE_TARDINESS == 1 )
{
	if( pxCurrentTCB->xDeadline < xTickCount )
	{
		xTardiness += ( xTickCount - pxCurrentTCB->xDeadline ) 
			* pxCurrentTCB->xTaskWeight;
	}
}
#endif
\end{lstlisting}

The scheduler was tested on a single task set under overload condition. 
The results will be presented through three examples with different priority calculation heuristics.
The considered task set is given in the following table.
\begin{table}[H]
\begin{center}
\begin{tabular}{|
>{\columncolor[HTML]{FFFFFF}}c |c|c|c|}
\hline
   & \cellcolor[HTML]{FFFFFF}\textbf{$T_i$} & \cellcolor[HTML]{FFFFFF}\textbf{$c_i$} & \cellcolor[HTML]{FFFFFF}\textbf{$w_i$} \\ \hline
\textbf{$\tau_1$} & 8                         & 6                      & 0.5   \\ \hline
\textbf{$\tau_2$} & 5                         & 2                      & 1     \\ \hline
\end{tabular}
\end{center}
\end{table}
The total utilization factor for the given set equals:
\begin{equation*}
\sum_{i=1}^{N}\frac{c_i}{T_i} = \frac{6}{8} + \frac{2}{5} = 1.15.
\end{equation*}

The schedule for the following task set provided by the FreeRTOS default scheduler is shown in figure \ref{freertos_def}.
\begin{figure}[ht]
    \centering
    \includegraphics[width=1\textwidth]{images/freertos_default_1.pdf}
    \caption{A schedule produced by the default FreeRTOS scheduler.}
    \label{freertos_def}
\end{figure}
Both tasks are assigned equal priorities. 
Task $\tau_1$ is the first task to be executed, since it was last added to the ready list. 
The instances of task $\tau_1$ start executing every time task $\tau_1$ is unblocked: $t=8$, $t=16$, $t=24$, $t=32$.
Task $\tau_2$ instances are executed while task $\tau_1$ is in the blocked state, since the activation of task $\tau_2$ is ignored while task $\tau_1$ is executing.
Therefore, every instance of task $\tau_2$ completes after its deadline. 
The total tardiness of task $\tau_2$ equals 44.

If task weight values are taken into account, the task priorities need to be set in a way that a task with a higher weight value has a higher priority.
In this example, task weight values are scaled by 2 in order to get integer priority values.
Figure \ref{freertos_def_2} shows a schedule produced with task $\tau_2$ priority set to 2 and task $\tau_1$ priority set to 1. 
\begin{figure}[ht]
    \centering
    \includegraphics[width=1\textwidth]{images/freertos_default.pdf}
    \caption{A schedule produced by the default FreeRTOS scheduler with task weights taken into account.}
    \label{freertos_def_2}
\end{figure}
In this example, task $\tau_2$ instances execute as soon as possible because it has a higher priority.
Task $\tau_1$ executes after task $\tau_2$ is switched to blocked state.
Upon the activation of task $\tau_2$, a context switch is performed, giving advantage to task $\tau_2$.
The total weighted tardiness of task $\tau_1$ instances equals 10.

The results of adding priority function to the scheduler will be presented through three examples.
In the first example, the task priority corresponds to the task deadline.
The result is shown in figure \ref{freertos_1}. 
\begin{figure}[ht]
    \centering
    \includegraphics[width=1\textwidth]{images/freertos_edf.pdf}
    \caption{A schedule produced by a modification of FreeRTOS scheduler, heuristic $p_i = d_i$.}
    \label{freertos_edf}
\end{figure}
Higher priority is given to the task instances with earliest deadline.
This heuristic is equivalent to the EDF algorithm.
Total tardiness of task $\tau_1$ is 8, while the $\tau_2$ tardiness equals 6.
The total weighted tardiness for this example equals 10.

In the second example, a heuristic evolved by genetic programming is used as the priority function.
The task's processing time is added to the priority function: $p_i = d_i + c_i$.
The produced schedule is depicted in Figure \ref{freertos_1}.
\begin{figure}[ht]
    \centering
    \includegraphics[width=1\textwidth]{images/freertos_heur1.pdf}
    \caption{A schedule produced by a modification of FreeRTOS scheduler, heuristic $p_i = d_i + c_i$.}
    \label{freertos_1}
\end{figure}
Priority ties are broken in favor of the task that is already executing.
An example of such situation is time instant $t=15$ where both task $\tau_1$ and $\tau_2$ have the same priority value, $22$.
As total weighted tardiness value is used as a penalty function for heuristics evolution, the produced schedule results in no late instances of the task with greater weight value.
The total tardiness of task $\tau_1$ equals 14, so the total weighed tardiness for this example is 7.

In the last example, a more complex heuristic is used as the priority function.
The task priorities are computed according to the expression:
\begin{align*}
p_i = d_i - SL \cdot \frac{c_i}{w_i}.
\end{align*}
Task priority is proportional to due date, but it takes the positive slack and weight value into account.
Advantage is given to the tasks with earliest due date and gratest positive slack value.
The result is shown in Figure \ref{freertos_2}.
\begin{figure}[ht]
    \centering
    \includegraphics[width=1\textwidth]{images/freertos_heur2.pdf}
    \caption{A schedule produced by a modification of FreeRTOS scheduler, heuristic $p_i = d_i - SL \cdot c_i / w_i$.}
    \label{freertos_2}
\end{figure}
Again, the task of greater weight value has no late instances.
The total tardiness of task $\tau_1$ equals $16$, thus the total weighted tardiness for this example equals $8$.
\chapter*{Conclusion}
This thesis presented a method for optimizing the skip factor of skippable periodic tasks.
The considered task model is typical for multimedia applications, where a dynamic scheduler is required.
Furthermore, the scheduler must be able to handle overload conditions and provide frequent schedule modifications.
Formal algorithms for scheduling skippable tasks (i.e. RLP and RLP/T) bring a significant overhead when applied to large amount of tasks, due to their computational complexity \cite{onqos}.
For that reason, heuristic scheduling is more acceptable in real-time systems with timeliness constraints due to its simplicity and ability to build a schedule with no overhead.
Genetic programming is suitable for evolving a heuristic which represents a priority function used for building a schedule.

Schedules built by evolved heuristics were compared to the RLP schedules with respect to QoS metric.
In overload conditions, the results of the evolved heuristic surpass the results achieved by RLP algorithm.
For utilization factor of $1.6$, the schedule built by the heuristic achieved a $30\%$ greater QoS value.

However, heuristic scheduling does not guarantee a minimum number of completed task instances between skipped ones.
In other words, skipping consecutive task instances is allowed, which can totally degrade system performance.
Therefore, using evolved heuristics for scheduling is applicable only in situations where task set parameters are known beforehand.

\newcommand{\namesigdate}[2][5cm]{%
  \begin{tabular}{@{}p{#1}@{}}
    #2 \\[2\normalbaselineskip] \hrule \\[15pt]
  \end{tabular}}

\vspace*{\fill} \noindent \hfill \namesigdate{Karla Salamun}

\bibliography{literatura}
\bibliographystyle{fer}

\begin{abstract}
This thesis presents the possibilities of using machine learning for scheduling tasks in overloaded real-time systems.
The existing formal algorithms for scheduling in overloaded real-time systems are investigated, as well as the methods for reducing system load in transient and permanent overload conditions.
A framework for evolving scheduling heuristics using genetic programming is implemented.
Support for multi-objective optimization and cooperative coevolution is integrated in the framework.
The performance of the evolved heuristics is measured by a simulator of periodic tasks execution with preemptive scheduling.
The interface for generating task sets for training and testing is implemented using the \textit{UUniFast} algorithm.
The results of testing the considered technique are compared to the results of the \textit{Red Tasks as Late as Possible} (RLP) algorithm with respect to Quality of Service (QoS) measure.
An existing framework for evolutionary computation is applied to the task scheduling problem in order to validate the results achieved by the implemented framework.
The FreeRTOS operating system is modified to use evolved heuristics for task scheduling and the results are evaluated on sythetically generated task sets.

\keywords{genetic programming, task scheduling, real-time systems.}
\end{abstract}

\newpage
% TODO: Navedite naslov na hrvatskom jeziku.
\hrtitle{Primjena genetskog programiranja za raspoređivanje zadataka u sustavima za rad u stvarnom vremenu}
\begin{sazetak}
U ovom radu prikazane su mogućnosti primjene strojnog učenja za raspoređivanje zadataka u preopterećenim sustavima za rad u stvarnom vremenu.
Istraženi su postojeći formalni algoritmi za raspoređivanje u preopterećenim sustavima za rad u stvarnom vremenu, kao i metode za smanjenje opterećenja u tranzijentnim i permanentnim uvjetima preopterećenja.
Implementiran je radni okvir za evoluciju heuristika za raspoređivanje koristeći genetsko programiranje.
U implementirani radni okvir integrirana je potpora za višekriterijsku optimizaciju i kooperativnu koevoluciju.
Rezultati evoluiranih heuristika analizirani su pomoću simulatora izvođenja periodičkih zadataka koji koristi raspoređivanje s istiskivanjem.
Sučelje za generiranje setova zadataka za učenje i testiranje implementirano je koristeći \textit{UUniFast} algoritam.
Rezultati testiranja predstavljene tehnike uspoređeni su s rezultatima algoritma \textit{Red Tasks as Late as Possible} (RLP) s obzirom na postignutu kvalitetu usluge.
Rezultati implementiranog radnog okvira verificirani su usporedbom s rezultatima postojećeg radnog okvira za evolucijsko računanje.
Napravljena je modifikacija operacijskog sustava FreeRTOS za korištenje evoluiranih heuristika za raspoređivanje zadataka.
Rezultati su evaluirani na sintetski generiranim skupovima zadataka.

\kljucnerijeci{genetsko programiranje, raspoređivanje zadataka, sustavi za rad u stvarnom vremenu.}
\end{sazetak}

\end{document}


 \begin{thebibliography}{1}

  \bibitem{cb1998} Caccamo, M. Buttazzo, G. {\em Optimal Scheduling for Fault-Tolerant and
Firm Real-Time Systems}  1998.

  \bibitem{impj}  The Japan Reader {\em Imperial Japan 1800-1945} 1973:
  Random House, N.Y.

  \bibitem{norman} E. H. Norman {\em Japan's emergence as a modern
  state} 1940: International Secretariat, Institute of Pacific
  Relations.

  \bibitem{fo} Bob Tadashi Wakabayashi {\em Anti-Foreignism and Western
  Learning in Early-Modern Japan} 1986: Harvard University Press.

  \end{thebibliography}