\documentclass[utf8, diplomski, english, numeric]{fer}
\usepackage{booktabs}
\usepackage{longtable}
\usepackage{enumitem}
\usepackage{algorithm}
\usepackage{subcaption}
\usepackage[export]{adjustbox}
\usepackage[noend]{algpseudocode}
\algnewcommand\And{\textbf{and}}
\algnewcommand\To{\textbf{to}}
\usepackage{verbatim}
\usepackage{float}
\usepackage[table,xcdraw]{xcolor}
\usepackage{pdfpages}

\usepackage{listings}
\lstset{
    language=C,
    basicstyle=\fontsize{11}{13}\ttfamily,
    % numbers=left,
    frame=single,
    % numberstyle=\tiny,
    showstringspaces=false,
    tabsize=4
}

\usepackage{color}
\definecolor{bluekeywords}{rgb}{0.13,0.13,1}
\definecolor{greencomments}{rgb}{0,0.5,0}
\definecolor{redstrings}{rgb}{0.9,0,0}

\newtheorem{mydef}{Definition}

\begin{document}

% TODO: Navedite broj rada.
\thesisnumber{1982}

% TODO: Navedite naslov rada.
\title{Application of Genetic Programming for Real-Time Task Scheduling}

% TODO: Navedite svoje ime i prezime.
\author{Karla Salamun}

\maketitle

% Ispis stranice s napomenom o umetanju izvornika rada. Uklonite naredbu \izvornik ako želite izbaciti tu stranicu.
\includepdf[pages={1-}]{hr_0036492033_48.pdf}

% Dodavanje zahvale ili prazne stranice. Ako ne želite dodati zahvalu, naredbu ostavite radi prazne stranice.
\zahvala{}

\tableofcontents
\listoffigures
% \lstlistoflistings

\chapter{Introduction}
\chapter{Genetic Programming}
\label{genprog}
\section{Optimization}
An optimization problem is the problem of finding the best solution to some problem instance. % from all feasible solutions.
Depending on the number of objective functions, optimization problems are distinguished in two categories: single-objective and multi-objective optimization problems.

\subsection{Single-objective optimization}
A single-objective optimization problem in its general form is defined in the following way:
\begin{align*}
\label{opt_problem}
\text{minimize/maximize } & f\_(\textbf{x}),   \\
\text{subject to } & g_j(\textbf{x}) \geq 0, & j = & 1, 2, ..., J \\
           & h_k(\textbf{x}) = 0, & k = & 1, 2, ..., K \\
           & x_{i}^{L} \leq x_i \leq x_{i}^{U}, & i = & 1, 2, ..., n. 
\end{align*}
Function $f(\textbf{x})$ is the objective function for the problem.
A solution $\textbf{x}$ is a vector of $n$ decision variables: $\textbf{x} = \{x_1,x_2,...,x_n\}$.
The set of $J$ functions $g_j(\textbf{x})$ is called inequality constraints, while the set of $K$ functions $h_k(\textbf{x}$ is reffered to as equality constraints.
In order for a solution to be feasible, all constraints, specifed with constraint functions and their respective bounds, have to be satisfied.
The last set of constraints defines a lower $x_{i}^{L}$ and an upper $x_{i}^{U}$ bound for the values that each decision variable. 
The described sets of constraints define a decision space $D$.
Each solution \textbf{x} is a single point in the decision space.
% The objective function $f$ to be maximized or minimized is often reffered to as the \textit{fitness} function.
Maximization problems can be reduced to minimization problems and vice versa by translating the value of the objective function to its complementary value, $-f$.
Thus, the greater value of a fitness function corresponds to a better solution.

\begin{mydef}
A solution $\textbf{x}^{*}$ is a global optimum if the solution $\textbf{x}^{*}$ is feasible and if $f(\textbf{x}^{*}) \geq f(\textbf{x})$ for every other solution $\textbf{x}$ from the feasible solution space.
\end{mydef}
If the function $f$ is not strictly convex, the existance of one or more local optima is possible.
\begin{mydef}
Solution $\textbf{x}^{*}$ is a local optimum if and only if the solution $\textbf{x}^{*}$ belongs to the feasible solution space and if for every other solution $\textbf{x}$ such that 
$\delta > 0, |\textbf{x} - \textbf{x}^{*}| \leq \delta$, the condition $f(\textbf{x}^{*}) \geq f(\textbf{x})$ is fulfilled.
\end{mydef}

If an optimization algorithm deterministically chooses the better solution than the current solution, the solutions generally converge towards the local optimum and the algorithm yields the wrong result.
A possible solution to this problem is stochastically accepting the solutions that are worse than the current solution.

\subsection{Multi-Objective Optimization}
A multi-objective optimization problem consists of multiple objective functions which are maximised or minimised. 
The general form of a multi-objective optimization problem is stated as follows:
\begin{align*}
\text{minimize/maximize } & f_m(\textbf{x}),  & m = & 1, 2, ..., M \\
\text{subject to } & g_j(\textbf{x}) \geq 0, & j = & 1, 2, ..., J \\
           & h_k(\textbf{x}) = 0, & k = & 1, 2, ..., K \\
           & x_{i}^{L} \leq x_i \leq x_{i}^{U}, & i = & 1, 2, ..., n. 
\end{align*}

While optimizing a set of $M$ functions, the problem can be stated in a way that some functions need to be maximized, while the others need to be minimized. 
Since the minimization problem of a function $f_i$ can be reduced to maximization of the function $-f_i$, the general form can be stated as a problem of maximization or minimization of all $M$ functions.

In multi-objective optimization, the objective functions constitute a multi-dimensional space $Z$, called the objective space. 
For each solution $\textbf{x}$ in the decision variable space, there exists a point in the objective space, denoted by $f(\textbf{x}) = \textbf{z} = \{z_1, z_2, ... z_M\}$ 
\cite{deb2001multi}. 
Since $\textbf{z}$ is a vector, two solutions 
$\textbf{x}^{(1)}$ and $\textbf{x}^{(2)}$ with corresponding vectors $\textbf{z}^{(1)}$ and $\textbf{z}^{(2)}$ cannot be directly compared. 

In most multi-objective optimization algorithms, comparison between solutions is done with the aid of the domination concept. 

\begin{mydef}
A solution $\textbf{x}^{(1)}$ is said to dominate the other solution $\textbf{x}^{(2)}$, if both conditions 1 and 2 are true: 
\begin{enumerate}
    \item The solution $\textbf{x}^{(1)}$ is no worse than $\textbf{x}^{(2)}$ in all objectives, or $z_j^{(1)} \geq z_j^{(2)}, \forall j \in {1,...,M}$.
    \item The solution $\textbf{x}^{(1)}$ is strictly better than $\textbf{x}^{(2)}$ in at least one objective, or $\exists j \in {1,...,M}: z_j^{(1)} > z_j^{(2)}$.
\end{enumerate}
\end{mydef}
There are three possibilities that can be the outcome of the dominance check between two solutions:
\begin{itemize}
    \item solution 1 dominates solution 2, 
    \item solution 1 gets dominated by solution 2,
    \item solutions 1 and 2 do not dominate each other.
\end{itemize}

In optimization algorithms, dominance is used to determine the quality of the solutions. 
By performing the dominance check between every pair of solutions, the solution set can be divided into two subsets: the solutions that are not dominated by other solutions and the solutions that get dominated by at least one solution. 
A set of all feasible solutions that are not dominated by any other solution is called a globally Pareto-optimal set. 

Some multi-objective optimization algortihms require sorting the population in different subsets with respect to their closeness to the non-dominated subset \cite{cupic2013prirodom}. 
Such subsets are referred to as fronts. 

A method for sorting the population into fronts is called non-dominated sorting. 
The complexity of the basic approach for non-dominated sorting is $\mathcal{O}(MN^3)$, where $M$ is the number of objective functions and $N$ is the population size.
An approach for non-dominated sorting used in this work is described with Alg. 
\ref{nondomsort}. This is a faster and more acceptable method for dominated sorting which requires at most $\mathcal{O}(MN^2)$ computations \cite{deb2001multi}.
For every solution $i$, a domination count $\mu_i$ is calculated. 
The domination count is the number of solutions which dominate the solution $i$. 
The solutions that have $\mu_i = 0$ form a non-dominated set.
Secondly, for every solution a set $S_i$ is determined. The set $S_i$ consists of the indices of all of the solutions that are dominated by the solution $i$. 
The solutions with $\mu_i=0$ are set as the first front. 
The next part of the process is to compute subsequent fronts.
This is done by finding the dominated solutions in the remaining population for each solution in the last computed front.
The counters $\mu_i$ for the dominated solutions are reduced. 
If for any solution the counter $\mu_i$ becomes zero, it means that the solution is not dominated by any solutions in the remaining population and it is added to a new front. 

\newpage
\begin{algorithm}
\caption{Non-dominated sorting.\label{nondomsort}}
\begin{algorithmic}[1]
\ForAll{$i \in P$}
\State set $\mu_i = 0$ and $S_i = \emptyset$
\EndFor
\ForAll{$j \neq i$ and $j \in P$}
\If{solution $i$ dominates solution $j$}
\State $S_i = S_i + j$
\ElsIf{solution $j$ dominates solution $i$}
\State increment counter $\mu_i = \mu_i + 1$
\EndIf
\If{$\mu_i = 0$} 
\State keep solution $i$ in the first non-dominated front $P_i$
\State set front counter $k=1$
\EndIf
\EndFor
\While{$P_k \neq \emptyset$}
\State initialize $Q = \emptyset$ which keeps the next non-dominated front
\ForAll{$i \in P_k \And j \in S_i$}
\State $\mu_j = \mu_j - 1$
\If{ $\mu_j = 0$} 
\State add solution $j$ to current front: $Q = Q + j$
\EndIf
\EndFor
\State set $k = k+1$
\State set $P_k = Q$
\EndWhile
\end{algorithmic}
\end{algorithm}

\section{Genetic Algorithms}

The algorithms that search the solution space to find good solutions, but do not guarantee finding the optimal solution are called \textit{heuristics}. 
Heuristics typically have relatively low computational complexity \cite{cupic2013prirodom}.
A \textit{metaheuristic} can be defined as an upper level general methodology (template) that can be used as a guiding strategy in designing underlying heuristics to solve specific optimization problems \cite{talbi2009metaheuristics}.

Stohastic optimization algorithms inspired by processes in biological evolution are called \textit{evolutionary algorithms}.
Besides stochastic optimization, evolutionary algorithms have a metaheuristic character.
In this context, a solution to a given optimization problem is called an \textit{individual} and a set of solutions is called a \textit{population}.
The basic approach in all evolutionary algorithms is applying evolutionary operators (selection, crossover, mutation and recombination) to a population of solutions.
Every interation of the algorithm corresponds to a generation.
As a result, the population gradually evolves towards better areas of the search space. 
The value which indicates how well the solution fulfills the problem objective is referred to as \textit{fitness} value.

Evolutionary algorithms are divided into four classes: 
\begin{itemize}
    \item genetic algorithms,
    \item genetic programming,
    \item evolutionary strategies,
    \item evolutionary programming.
\end{itemize}

The general approach in genetic algorithms is to generate a population of genotypes and apply evolution operators on them.
A genotype is a data structure which represents a potential solution to the given problem.

Every genotype is assigned a fitness value by computing the value of the objective function in the point represented by the genotype.
This process is called evaluation.
In every generation, some genotypes are selected from the population by a defined selection operator which favors the best solutions against the worst ones with respect to the fitness function.
The selected members are referred to as parents.
The crossover operator is applied to the parents, creating the children genotypes.
In that way, the recombination of genes is emulated - two or more parts from parent genotypes are combined to form an offspring.
After crossover, the mutation operator is applied to the offspring.
Mutation randomly modifies a single solution.
The algorithm cycle is finished by adding the children to the population by reinsertion operator.

There are two possible implementations of the genetic algorithm: the steady-state genetic algorithm and the generational genetic algorithm.

A schematic view of the steady-state implementation is shown in Fig. \ref{genalg:impl_1}.
\begin{figure}[ht]
    \centering
    \includegraphics[width=1\textwidth]{images/genalg_impl_1.pdf}
    \caption{A schematic view of the steady-state genetic algorithm.}
    \label{genalg:impl_1}
\end{figure}
In every generation, two parents are selected.
The offspring is created by performing crossover on the parents and mutating the resulting genotype.
Before inserting into the population, the offspring is evaluated to determine whether it can replace an existing population member.
If the offspring is inserted in the population, it usually replaces the worst genotype in the population.
Alternatively, the genotype to be replaced can be the oldest member, a randomly selected member or it can be determined by a selection operator modified in a way that it prefers selecting worse genotypes.
A pseudocode for the steady-state genetic algorithm is given in Alg. \ref{genalg_ss}.
\begin{algorithm}
\caption{Steady-state genetic algorithm.\label{genalg_ss}}
\begin{algorithmic}[1]
\State $\text{population} \gets create\_initial\_population()$
\State $evaluate(\text{population})$
\While{$! \; stop\_condition$}
\State $\text{parents} \leftarrow selection(\text{population})$
\State $\text{offspring} \leftarrow crossover(\text{parents})$
\State $mutate(\text{offspring})$
\State $evaluate(\text{offspring})$
\State select a genotype to be replaced by offspring
\State decide whether replacement will be performed, replace members if needed
\EndWhile
\end{algorithmic}
\end{algorithm}

In the generational genetic algorithm, in every generation a new offspring population is created, which completely replaces the parent population.
This implementation is shown in Fig. \ref{genalg:impl_2}.
\begin{figure}[ht]
    \centering
    \includegraphics[width=1\textwidth]{images/genalg_impl_2.pdf}
    \caption{A schematic view of the generational genetic algorithm.}
    \label{genalg:impl_2}
\end{figure}
In every step of the algorithm, a new temporary population is created, consisting only of the offspring members.
When the size of the new population reaches the size of the parent population, the latter is replaced by the new population.
The basic approach to generational genetic algorithm is not elitistic, i.e. it does not necessarily keep the best solution found. 
A modification of the algorithm that involves elitism directly adds the current best solution (or a few of them) to the offspring population.
The rest of the offspring population is generated by applying crossover on the parents and mutating the result.
An elitistic implementation of the generational genetic algorithm is given in Alg.
 \ref{genalg_g}.

\section{Tree-Based Genetic Programming}
Genetic programming is an extension of the conventional genetic algorithm where the genotypes represent a program that solves the considered problem.
This is achieved by increasing the complexity of the structures undergoing adaptation.
A typical data structure used in genetic programming is a tree which represents a hierarchical computer program of dynamically varying size and shape \cite{koza1992genetic}.
Fig. \ref{genprog:ex1} depicts an example of such data structure which can be represented as a function:
\begin{equation*}
f(x, y, z) = 2 \cdot (x-5) + \frac{y}{z + 3} \, .
\end{equation*}

\begin{algorithm}
\caption{Elitistic variant of the generational genetic algorithm.\label{genalg_g}}
\begin{algorithmic}[1]
\State $\text{population} \gets create\_initial\_population()$
\State $evaluate(\text{population})$
\While{$! \; stop\_condition$}
\State $\text{new\_population} = \emptyset$
\State $\text{new\_population} += get\_best(\text{population})$
\While{$\text{new\_population.size()} < \text{population.size()}$}
\State $\text{parents} \gets selection(\text{population})$
\State $\text{offspring} \leftarrow crossover(\text{parents})$
\State $mutate(\text{offspring})$
\State $\text{new\_population} += \text{offspring}$
\EndWhile
\State $\text{population} \leftarrow \text{new\_population}$
\EndWhile
\end{algorithmic}
\end{algorithm}

\begin{figure}
    \centering
    \includegraphics[width=0.7\textwidth]{images/gp_ex1.pdf}
    \caption{An example of a tree structure representing a program.}
    \label{genprog:ex1}
\end{figure}

Each genotype is composed of function and terminal nodes.
Function nodes may be standard arithmetic operations, programming operations, mathematical or logical functions.
In Fig. \ref{genprog:ex1}, function nodes are arithmetical operators ($+, -, *, /$), while the other nodes are terminal.
Each terminal node represents a parameter related to the specific problem and can be of various data types.
Moreover, a terminal node can be a constant ($2, 5, 3$ in Fig. \ref{genprog:ex1}) or a variable ($x, y, z$).
As in general genetic algorithms, each individual is assigned a fitness value in the evaluation process.
The computation of the fitness value depends on the nature of the problem.
In general, fitness value measures how well the genotype solves the given problem.

The initial population consists of random compositions of the function and terminal nodes.
The evolution is performed in the following way \cite{koza1992genetic}.
Each program (genotype) in the population is executed and is assigned a fitness value according to how well it solves the problem.
New population is generated by copying a part of the existing genotypes to the new population, while the rest is created by genetically recombining randomly chosen parts of parent genotypes.
The result of genetic programming is the best individual that appeared in any generation.

There are various selection mechanisms proposed in the literature, with different amount of selection pressure (or \textit{greediness}).
As greedy algorithms consider selecting only the best individuals, they have the general property of rapid convergence, but not necessarily to the best solution.
For some problem domains, convergence to local optima is acceptable, but other applications require a slower and more diffuse search \cite{rozenberg2012handbook}.
The most common selection methods include the following, in the order of decreasing selection pressure:
\begin{itemize}
	\item truncation selection: choose only the N best individuals,
	\item tournament selection: choose K individuals at random and keep only the best of the K selected,
	\item rank-proportional selection: choose individuals proportional to their fitness rank in the population,
	\item fitness-proportional selection: choose individuals proportional to the value of their fitness.
\end{itemize}
In this work, the tournament selection is used.

Crossover operator is performed as replacing one parent's subree with a randomly chosen subtree of the other parent.
This is illustrated in Fig. \ref{genprog:ex2}.
\begin{figure}[ht]
    \centering
    \includegraphics[width=0.9\textwidth]{images/crossover.pdf}
    \caption{Crossover operator in genetic programming.}
    \label{genprog:ex2}
\end{figure}

Mutation of the offspring can be performed in multiple ways: by replacing a node by a randomly generated subtree, erasing a subtree or a node, rotation of the child nodes etc.
An example of a mutated individual is shown in Fig. \ref{genprog:ex3}.
\begin{figure}[H]
    \centering
    \includegraphics[width=1\textwidth]{images/mutation_1.pdf}
    \caption{Mutation operator in genetic programming.}
    \label{genprog:ex3}
\end{figure}

\section{Multi-objective Optimization using Genetic Algorithms}
Genetic algorithm based on the concept of non-dominated sorting is the Non-Dominated Sorting Genetic Algorithm (NSGA). 
In the evaluation of the solutions, each individual is assigned a scalar value which represents its fitness. 
The individuals that belong to a front closer to the non-dominated set are assigned a greater fitness value. Additionally, the individuals of the same front with clustered objective values are assigned a diminished fitness value. 
In that way, the diversity between the individuals is stimulated. 

In this work, the algorithm NSGA-II is used for handling multi-objective optimization problems. 
NSGA-II is an improved version of NSGA algorithm and it will be described in the following subsection. 

\subsection{NSGA-II Algorithm}
One of the main disadvantages of the original NSGA algorithm is the lack of elitism which significantly slows down the performance of the genetic algorithm due to the loss of good solutions once they have been found \cite{deb2000fast}. 
The NSGA-II algorithm is a modification of the original NSGA algorithm and its main feature is elitist approach in selecting new population members. 
An implementation of the NSGA-II algorithm is described with Alg. \ref{nsga-ii}.

The NSGA-II algorithm creates an offspring population $Q_t$ of the same size $N$ as the parent population $P_t$.
A combined population $R_t = P_t \cup Q_t$ of size $2N$ is formed.
The population $R_t$ is then sorted according to non-dominated sorting. 
After non-dominated sorting, the population is fragmented into $\rho$ fronts: $\mathcal{F}_1, ..., \mathcal{F}_\rho$.
The new parent population $P_{t+1}$ is formed by adding solutions from the fronts.
Since the size of the new population must be equal to $N$, the fronts are added to the new population in increasing rank order, as long as they can be fully added.

An additional benefit of NSGA-II is the perservation of the diversity between individuals. 
Therefore, the inidividuals that are added to the new population from the last remaining front are determined according to their diversity. 
For finding the most "scattered" solutions, the crowding-sort method is used. 
The density of the solutions surrounding a particular point in the population is estimated by crowding distance. 
In the case of a two-objective problem, this value refers to the average distance of the two points on either side of the particular point along each of the objectives, as shown in Fig. 
\ref{crowding_dist}. 

\begin{figure}[ht]
    \centering
    \includegraphics[width=0.50\textwidth]{images/crowd_dist.pdf}
    \caption{Crowding distance. The crowding distance of the $i$-th solution in its front (marked with solid circles) is the average side-length of the cuboid (marked with dashed lines). }
    \label{crowding_dist}
\end{figure}

Crowding distance $i_{distance}$ is an estimate of the size of the largest cuboid enclosing the point $i$ without including any other point in the population. 
The crowding-sort method arranges the individuals according to the crowding distance value.

After creating the new parent population, an offspring population is created by crowding tournament selection, crossover and mutation. 
Crowding tournament selection is based on the individual's rank (the number of the associated front) and the crowding distance. 
According to the crowding tournament selection, solution $i$ is the tournament winner if any of the following conditions are fulfilled:
\begin{enumerate}
	\item solution $i$ has a better rank than solution $j$: $r_i < r_j$,
	\item solution $i$ has an equal rank as solution $j$, but has a larger crowding distance, i.e. $r_i=r_j, d_i > d_j$.
\end{enumerate}

\begin{algorithm}
\caption{NSGA-II algorithm.\label{nsga-ii}}
\begin{algorithmic}[1]
\State combine the current parent population $P_t$ and offspring population $Q_t$ into population $R_t$, $R_t = P_t \cup Q_t$
\State perform non-dominated sorting of the population $R_t$, which creates $\rho$ fronts: 
$\mathcal{F}_1, ..., \mathcal{F}_\rho$ 
\State set the new parent population to an empty set: $P_{t+1} = \emptyset$
\State set $i=1$
\While{front $i$ can fit completely into population $P_{t+1}$}
\State add current front to $P_{t+1}$: $P_{t+1} = P_{t+1} \cup \mathcal{F}_i$
\State $i=i+1$
\EndWhile
\State perform grouping sort of the front $\mathcal{F}_i$
\State add $N-|P_{t+1}|$ solutions from the front $\mathcal{F}_i$ to $P_{t+1}$
\State perform crowded tournament selection of the population $P_{t+1}$
\State create new offspring population $Q_{t+1}$ by crossover and mutation
\end{algorithmic}
\end{algorithm}

\section{Cooperative Coevolution}
Coevolutionary architectures of genetic algorithms define the fitness evaluation of an individual in relation and adaptation with respect to the other individuals in the population. 
Coevolution can be cooperative or competitive. 
In competitive coevolution, the evaluation of an individual is determined by a set of competitions between itself and other individuals. 
Contrarily, in cooperative coevoluiton, collaborations between a set of individuals are necessary in order to evaluate one complete solution \cite{stoean2014support}. 
Since cooperative coevolution is of interest in this work, this model will be described. 

In cooperative coevolution, two or more genetically isolated species cooperate in solving a target problem. 
Genetic isolation is enforced by evolving the species in separate populations. 
The species exchange information only when the individuals are evaluated. 
Target problem statement is decomposed into components and each subproblem is assigned to a species. 
The fitness of each member is evaluated by forming collaborations with individuals from other species. 

An application of cooperative coevolution is optimization of functions with multiple variables. 
Each variable is considered as a component of the solution vector and corresponds to a single population. 

The general cooperative coevolutionary algorithm is described with Alg. \ref{coop}.

\begin{algorithm}
\caption{Cooperative coevolution algorithm.\label{coop}}
\begin{algorithmic}[1]
\State set $t=0$
\ForAll{species $s$}
\State randomly initialize population $P_s(t)$
\EndFor
\ForAll{species $s$}
\State evaluate $P_s(t)$ by choosing collaborators from the other species
\EndFor
\While{termination condition is not satisfied}
\ForAll{species $s$}
\State select parents from $P_s(t)$
\State apply crossover and mutation
\State evaluate offspring by choosing collaborators from the other species
\State select new population $P_s(t+1)$
\EndFor
\State set $t=t+1$
\EndWhile
\end{algorithmic}
\end{algorithm}

First, each population is initialized.
In the first evaluation, random individuals from each of the other populations are selected and obtained solutions are evaluated. 
Next, each population is evolved in the same way as specified by a regular genetic algorithm.
For further evaluations, three attributes are considered when selecting collaborators 
\cite{stoean2014support}:
\begin{itemize}
	\item collaborator selection pressure, 
	\item collaboration pool size,
	\item collaboration credit assignment.
\end{itemize}
Collaborator selection pressure refers to the way of selecting collaborators from other populations when evaluating an individual. 
One can select the best individual according to its previous fitness score, pick a random individual or use classic selection schemes. 
Collaboration pool size represents the number of collaborators which are selected from each population. 
Collaboration credit assignment determines the way of computing the fitness of the current individual. This attribute is considered when the collaboration pool size is higher than 1. 
There are three possibilities for calculating the fitness value based on multiple collaborators:
\begin{itemize}
	\item optimistic: the fitness of the current individual is the value of its best collaboration,
	\item hedge: the fitness of the current individual is the average value of its collaborations,
	\item pessimistic: the fitness of the current individual is the value of its worst collaboration. 
\end{itemize}
An individual is evaluated by forming a number of collaborations according to the collaboration pool size. The collaborators are selected through a strategy corresponding to the collaboration selection pressure. 
The fitness value of an individual is measured as a result of its collaborations.
A way of assigning the fitness value to an individual based on its collaborations is defined by the collaboration credit assignment.

\chapter{Formal Methods of Scheduling Overloaded Real-Time Systems}
\section{General approach for task scheduling}
The timing behavior of a real-time system is analyzed by modelling all software activities running in the processor as a set of $n$ real-time tasks: $T = \{\tau_1, \tau_2, ..., \tau_n\}$.
A task $\tau_i$ is considered as an infinite sequence of instances, or jobs $\tau_{i,j}$.
A task is said to be periodic if all its jobs are released one after the other with an interval $T_i$ called the task period.
% Each job has a computation time $c_{i,j}$, a release time $r_{i,j}$ and an absolute deadline $d_{i, j}$\cite{lee2007handbook}.
% For simplicity, all jobs of the same task are assumed to have the same computation time and relative deadline, which will be denoted in the following text as $c_i$ and $d_i$, respectively.

The operation of allocating the CPU to a selected task is referred to as \textit{dispatching}.
A set of rules that determines the selection of a task to be executed at current time is called a 
\textit{scheduling algorithm}.
The scope of this work considers scheduling various sets of concurrent periodic tasks which execute on a single processor.
The formal definition of a schedule is stated in the following text.
\begin{mydef}
A schedule is defined as a function $\sigma : \mathbb{R}^{+} \rightarrow \mathbb{N}$ such that 
$\forall t \in \mathbb{R}^{+}, \exists \; t_1, t_2$ such that $t \in [\, t_1, t_2 \rangle$ and 
$\forall \; t' \; \exists \; [\, t_1, t_2 \rangle \; \sigma(t) = \sigma(t')$.
\end{mydef}
In other words, $\sigma(t)$ is an integer step function and $\sigma(t) = k$, with $k > 0$, means that task $J_k$ is is executing at time $t$, while $\sigma(t) = 0$ means that the CPU is idle.
An example of a schedule function is shown in figure \ref{schedule_function}.
\begin{figure}[ht]
    \centering
    \includegraphics[width=1\textwidth]{images/sched_function.pdf}
    \caption{Schedule obtained by executing tasks $J_1$, $J_2$ and $J_3$. Modified from \cite{buttazzo2011hard}.}
    \label{schedule_function}
\end{figure}

A schedule is said to be \textit{feasible} if all tasks can be completed according to a set of specific constraints.
A set of tasks is said to be \textit{schedulable} if there exists at least one algorithm that can produce a feasible schedule.

\subsection{Timing Constraints of Real-Time Tasks}
In this work, the timing constraints of the tasks are of interest, specifically the task deadlines.
A deadline represents the time before which a task should complete its execution without causing any damage to the system.
A deadline specified with respect to the task arrival time is called an \textit{absolute deadline}, whereas a deadline specified with resprect to time zero is called a \textit{relative deadline}.
The timing parameters characterizing a real-time task $\tau_i$ are the following:
\begin{itemize}
    \item arrival time $a_i$ - the time at which a task becomes ready for execution,
    \item computation time $c_i$ - the CPU time necessary for executing the task without interruption,
    \item absolute deadline $d_i$ - the time before which a task should be completed to avoid damage to the system,
    \item start time $s_i$ - the time at which a task starts its execution,
    \item finish time $f_i$ - the time at which a task finishes its execution.
\end{itemize}
The timing parameters of a task are illustrated in Figure \ref{task_timing}.
\begin{figure}[ht]
    \centering
    \includegraphics[width=0.8\textwidth]{images/task_timing.pdf}
    \caption{Timing parameters of a real-time task.}
    \label{task_timing}
\end{figure}
The task lateness represents the delay of a task completion with respect to its deadline:
\begin{align*}
L_i = f_i - d_i.
\end{align*}
If a task completes before the deadline, its lateness is negative.
A parameter which considers task completion after the deadline is the task tardiness - the amount of time a task stays active after its deadline:
\begin{align*}
E_i = max( 0, L_i ).
\end{align*}

Task importance is specified through a value parameter, associated with each task.
Depending on particular application, task value may be directly related to task parameters such as computation time or deadline value, but can also be set to an arbitrary value. 
The value associated with a task as a function of its completion time is called a \textit{utility function}.
Utility function is used to describe the benefit of executing a task depending on the time at which the task is completed in relation to the task deadline.
Depending on the consequences of a missed deadline, real-time tasks are usually distinguished in four categories \cite{buttazzo2011hard}:
\begin{itemize}
    \item non real-time tasks,
    \item soft real-time tasks,
    \item firm real-time tasks,
    \item hard real-time tasks.
\end{itemize}
The typical utility functions differ for the stated categories and are shown in Figure \ref{utility}.
\begin{figure}[ht]
    \centering
    \includegraphics[width=0.8\textwidth]{images/task_value.pdf}
    \caption{Utility functions for different categories of real-time tasks.}
    \label{utility}
\end{figure}
Non real-time tasks do not have deadlines. The task value is proportional to the task importance and it does not depend on the completion time.
For soft real-time tasks, the task value is constant if the task finishes before its deadline, but it decreases with the exceeding time. 
Thus, a soft deadline is not critical, but missing the deadline causes performance degradation.
In the case of firm real-time tasks, executing a task after its deadline brings no benefit for the system - the utility function is zero after the deadline. However, missing the deadline does not cause any damage to the system.
Hard real-time tasks must complete before their deadline. Missing the deadline may cause catastrophic consequences on the system.

\subsection{Classification of Scheduling Algorithms}
The algorithms for scheduling real-time tasks proposed in literature can be classified by the following criterions \cite{buttazzo2011hard}:
\begin{itemize}
    \item support for preemption,
    \item static / dynamic algorithms,
    \item offline / online usage,
    \item optimal / heuristic.
\end{itemize}
In preemtive algorithms, the running task can be interrupted at any time to assign the processor to another active task of higher priority.
On the other hand, in non-preemptive algorithms a task is executed until completion, without being interrupted. 
All scheduling decisions are taken as the task terminates its execution.
Static algorithms are those in which scheduling decisions are based on fixed parameters, assigned to tasks before their activation.
In dynamic algorithms, the task parameters are assigned dynamically during execution.
A scheduling algorithm is used offline if it is generated before tasks activation. 
The generated algorithm is stored and executed by a dispatcher.
Online scheduling algorithms make scheduling decisions every time a new task enters the system or when a running task terminates.
An algorithm is said to be optimal if it minimizes some given cost function defined over the task set. 
When no cost function is defined, an algorithm is optimal if it is able to find a feasible schedule.
An algorithm is heuristic if it is guided by a heuristic function in taking its scheduling decisions. 
It tends toward the optimal schedule, but it does not guarantee finding it \cite{buttazzo2011hard}.

\subsection{Task Management in Real-Time Systems}
A task competing for CPU time is called an \textit{active} task.
If an active task is currently executing, it is referred to as the \textit{running} task.
The other tasks waiting for CPU time are in the \textit{ready} state and are kept in the \textit{ready queue}.
Figure \ref{ready_queue} shows a schematic view of the ready queue.
\begin{figure}[ht]
    \centering
    \includegraphics[width=1\textwidth]{images/ready_queue.pdf}
    \caption{Queue of ready tasks waiting for execution. Modified from \cite{buttazzo2011hard}.}
    \label{ready_queue}
\end{figure}
In operating systems that allow dynamic activation, the running task can be interrupted at any moment if a more important task arrives in the system.
CPU is assigned to the most important ready task, while the running task is inserted in the ready queue.
This operation is called \textit{preemption}.
Preemption usually increases the efficiency of the schedule in a sense that it allows executing real-time task sets with higher processor utilization \cite{buttazzo2011hard}.

In standard queueing theory, system load $\rho$ represents the expected number of job arrivals per mean service time. 
In a system consisting of preemptable periodic tasks with implicit deadlines, system load is equivalent to the processor utilization factor:
\begin{equation*}
\rho = U = \sum_{i=1}^{n}\frac{C_i}{T_i},
\end{equation*}
where $C_i$ is the computation time and $T_i$ is the period of task $\tau_i$.
If load value is greater than one, the task set is non-schedulable because the total computation demand requested by the periodic tasks exceeds the available processor capacity.
Hence, not all tasks can complete within their deadlines.
This condition in real-time task scheduling is called an \textit{overload condition}.
System overload may occur due to bad system design, simultaneous arrival of "unexpected" events, malfunctioning of input devices, operating system exceptions and so on.
Methods of handling overload conditions presented in this chapter consider firm real-time tasks.

\section{Transient Overload}
Transient overload occurs for a limited duration, in a system with the average load less or equal to one.
Due to aperiodic requests or unexpected behaviour of some tasks, the maximum load can exceed one and cause a transient overload.

\subsection{Aperiodic Overloads}
This type of overload occurs in systems consisting of aperiodic jobs, due to excessive event arrivals. 
This type of overload can lead to a phenomenon called the \textit{Domino Effect} in which the arrival of a new task can cause all of the previous tasks to miss their deadlines. 
Figure \ref{aperiodic} depicts such situation on an EDF-scheduled task set. 
The execution of task $\tau_0$ leads to deadline miss of all the previous tasks.

\begin{figure}[ht]
    \centering
    \includegraphics[width=0.70\textwidth]{images/overload_aper.pdf}
    \caption{An example of transient overload due to aperiodic job arrival under EDF-scheduled system.}
    \label{aperiodic}
\end{figure}

As shown in \cite{buttazzo2011hard}, it is impossible to create an optimal on-line algorithm that handles overload situations.
Since aperiodic environment requires using an online scheduler, there is no algorithm that can guarantee a feasible schedule.
However, there are different online scheduling algorithms proposed in the literature and they can be divided into three main classes:
\begin{itemize}
    \item{best effort,}
    \item{with acceptance test,}
    \item{robust.}
\end{itemize}

Best effort algorithms always accept new tasks into the ready queue upon arrival. Tasks to be skipped are determined through proper priority assignment with respect to their value.

Algorithms with acceptance test perform a guarantee test at every job activation. 
A new task that enters the system is accepted if the task set is found schedulable by the guarantee routine. Otherwise, task is rejected.
If an acceptance test is performed, the system is able to avoid domino effects.
The disadvantage of acceptance tests is that they do not take task importance into account and always reject newly arrived tasks.

Robust algorithms separate the schedulability verification into two routines: one for task acceptance and one for task rejection. 
Task rejection routine solves the problem of rejection of the newly arrived tasks with high importance.

\subsection{Task Overruns}
In overrun condition, a task (or a job) exceeds its expected utilization value.
Overruns occur if a job computation time exceeds the expected value, or due to early activation of the next job.
Transient overloads due to overruns occur if tasks execute for a longer time than expected, or more frequently than expected.

Figure \ref{transient_EDF} shows the effect of transient overload condition under EDF-scheduled system. Task $\tau_2$ exceeded its expected computation time, consequently causing task $\tau_1$ to miss its deadline.
\begin{figure}[ht]
    \centering
    \includegraphics[width=0.8\textwidth]{images/tran_overload.pdf}
    \caption{An example of transient overload due to task overrun under EDF-scheduled system.}
    \label{transient_EDF}
\end{figure}

Overruns are prevented by either aborting the job experiencing an overrun or letting it continue with a lower priority. 
The first solution is not safe because the job could be in a critical section when aborted, thus leaving a shared resource with inconsistent data 
\cite{buttazzo2011hard}. 

A method for implementing the second solution is called resource reservation.
Each task is assigned a fraction of the processor bandwidth, just enough to satisfy its timing constraints, while the kernel prevents each task from consuming more than the requested amount \cite{buttazzo2011hard}. 
A general approach for implementing this method is to reserve each task an amount of processor time $Q_i$ in every time interval $P_i$. 
The reservations can be hard or soft: a hard reservation allows the reserved task to execute at most for $Q_i$ units of time every $P_i$, while a soft reservation allows it to execute for more than $Q_i$ units if there is idle time available. 
In this way, the tasks are effectively reshaped and they can be considered as 
periodic real-time tasks with parameters \( (Q_i, P_i) \) that can be scheduled by a classical real-time scheduler.
The disadvantage of this approach is that the system performance becomes dependent on correct resource reservation. The task may slown down the system if is assinged a much less CPU bandwidth than its average requested value. Contrarily, if the allocated bandwidth is greater than needed, the system will run with low efficiency.

%%%%%%%%%%%%%%%%%%%%%%%%%%%%%%%%%%%%%%%%%%%%%%%%%%%%%%%%%%%%%%%%%%%%%%%%%%%%%%%%%%%%%%%%%%%%%%
\section{Permanent Overload}

Permanent overload in periodic task systems occurs when the total utilization factor of the periodic task set is greater than one \cite{buttazzo2011hard}. 
This condition occurs either because of wrong estimation of the task execution time, unexpected activation of new periodic tasks, or the increase of activation rate of current tasks.
Methods for reducing permanent overload are the following:
\begin{itemize}
	\item{job skipping,}
	\item{period adaptation,}
	\item{service adaptation.}
\end{itemize}

\subsection{Job Skipping}
\label{skip_algs}
Job skipping method reduces system load by skipping some jobs in the task set. Jobs to be skipped are assigned in a way that the remaining jobs can be scheduled within their deadlines.
This method is applicable for firm real-time tasks, as they allow a certain miss ratio. 
The quantified ratio of tasks that may not be executed is directly related to Quality of Service 
(QoS) metric. 
The algorithms presented in this section have the purpose of maximizing the QoS of periodic tasks.

A task model suitable for job skipping method is known in literature as the firm periodic model.
It was first described in \cite{koren1995skip}.
According to this model, every task is described as following:
\begin{align*}
\tau_i(c_i, T_i, d_i, s_i)
\end{align*}
where $c_i$ is the worst-case computation time, $T_i$ is the task period, $d_i$ is the relative deadline (assumed equal to the period) and $s_i$ is the skip parameter which marks the minimum distance between two consecutive skips, \(2 \leq S_i < \infty\).

An example of schedule with task skipping is given in figure \ref{skipover}. 
Since the utilization factor equals 
\begin{equation*}
U_p = \sum_{i=1}^{n} \frac{C_i}{T_i} = 1.17,
\end{equation*}
the system is permanently overloaded. 
However, if task \(\tau_2\) is skipped every three instances, the overload is resolved.
\\
\begin{figure}[ht]
    \centering
    \includegraphics[width=1\textwidth]{images/skipover.pdf}
    \caption{An example of resolving permanent overload by skipping method.}
    \label{skipover}
\end{figure}

Corresponding to the stated task model, every job of a periodic task can be red or blue.
A red job must be completed within its deadline, whereas a blue job can be aborted at any time 
\cite{buttazzo2011hard}. 
With respect to the skip parameter $S_i$, each scheduling algorithm needs to fulfill two conditions:
\begin{itemize}
	\item if a blue job is skipped, then the next \(S_i - 1\) jobs must be red,
	\item if a blue job completes successfully, the next job is also blue.
\end{itemize}
The authors of the model proved that the problem of determining whether a set of periodic skippable tasks is schedulable is NP-hard \cite{koren1995skip}.

For the given set of skippable periodic tasks \( \Gamma = {\tau_i(C_i,T_i,S_i)} \), the necessary schedulability condition is the following:
\begin{equation}
\sum_{i=1}^{n} \frac{C_i(S_i-1)}{T_i S_i} \leq 1.
\end{equation}

A sufficient condition for guaranteeing schedulability of a set of skippable tasks can be stated 
using the equivalent utilization factor, which is formally described in the following definition:

\begin{mydef}
Given a set \( \Gamma = {\tau_i(C_i,T_i,S_i)} \) of n skippable periodic tasks, the equivalent utilization factor in time interval \([0,L]\) is defined as:
\begin{equation*}
U_p^\ast = \max_{L \geq 0} \frac{\sum_{i}D(i, [0,L])}{L}
\end{equation*}
where
\begin{equation*}
D(i, [0,L]) = \left\lfloor \frac{L}{T_i} - \frac{L}{T_i \cdot S_i}\right\rfloor \cdot C_i.
\end{equation*}
\end{mydef}

According to the sufficient schedulability condition, a set of skippable periodic tasks is schedulable if \( U_p^\ast \leq 1 \).

\subsubsection{RTO Algorithm}
The Red Tasks Only (RTO) algorithm decreases system load by always skipping the blue tasks, while red tasks are scheduled according to EDF. 
% The algorithm becomes optimal under a certain condition, the \textit{deeply-red} condition:
% all tasks are synchronously activated and the first \(S_i - 1\) instances of every task $\tau_i$ are red. 
% The stated condition is the worst-case condition for periodic task sets.
As all instances of blue tasks are systematically rejected rregardless of their impact on the equivalent utilization factor, RTO schedule has the lowest QoS level for given task set.
Implementation of RTO algorithm was tested on an example shown in 
figure \ref{rto}.
Total processor utilization $U_p$ equals 1.19 and the equivalent processor 
utilization $U_p^\ast$ equals 0.79.
As the schedule results in seven missed instances, QoS equals 0.53. 
\\
\begin{figure}[ht]
    \centering
    \includegraphics[width=1\textwidth]{images/skipover_RTO.pdf}
    \caption{An example of a schedule produced by RTO algorithm with skipping factor $s_i=2$.}
    \label{rto}
\end{figure}

\subsubsection{BWP Algorithm}
Blue When Possible (BWP) algorithm introduces amendments of RTO algorithm disadvantages in a way that blue tasks execute only if there are no red ready instances.
As BWP increases the number of task instances that complete successfully, it offers a higher QoS level. 
Figure \ref{bwp} depicts the same overload condition as described in figure 
\ref{rto}, but solved by BWP algorithm. 
Total number of completed task instances is increased because two blue instances completed successfully, thus increasing QoS to 0.67.
\\
\begin{figure}[ht]
    \centering
    \includegraphics[width=1\textwidth]{images/skipover_BWP.pdf}
    \caption{An example of a schedule produced by BWP algorithm with skipping factor $s_i=2$.}
    \label{bwp}
\end{figure}

\subsubsection{RLP Algorithm}
The Red Tasks as Late as Possible (RLP) algorithm stimulates the execution of blue tasks by executing them at idle times of a schedule considering only the red tasks. 
A basis of this approach is the Earliest Deadline as Late as Possible (EDL) algorithm. 
EDL is used to build a schedule on the red instances only, where red instances execute as late as possible. 
In the remaining EDL idle times, blue tasks are executed as soon as possible. 
The main idea of EDL is to maximize the length of idle time periods at the beginning of the schedule. 

Determination  of  the  duration  and  position  of  these  idle  times  is  done  by mapping out the EDL schedule produced from time zero up to the end of the first hyperperiod \cite{ghor2011real}.
This approach makes any spare processing time available as soon as possible, 
thus effectively stealing slack from the hard deadline periodic tasks 
\cite{queudet2012quality}. 
Available slack at any time can be determined by mapping out an EDL schedule from current time to the end of the current hyperperiod.
EDL schedule is constructed dynamically at run-time from a static EDL schedule.
A static EDL schedule is constructed off-line and determined by two vectors:
\begin{itemize}
	\item{$K$ - static deadline vector,}
	\item{$D$ - static idle time vector.}
\end{itemize}
Vector $K$ stores the time instants from 0 to the end of the first hyperperiod at which idle times occur. 
Vector $D$ represents the lengths of idle times starting at time instances stored in $K$. 
Dynamic EDL schedule takes the execution of current running task into account and it is described by the following vectors:
\begin{itemize}
	\item{$K_t$} - dynamic deadline vector,
	\item{$D_t$} - dynamic idle time vector.
\end{itemize}
$K_t$ represents the instants $k_i$ from $t$ in the current hyperperiod at which idle times occur. 
The lengths of idle times starting at instants $k_i$ are stored in the vector $D_t$. 

% As EDL idle time is determined by the described vectors, EDL schedule itself is not considered in this work because the vectors can be computed with the aid of EDF schedule.

In this work, EDL idle times are computed through corresponding EDF values, according to the expression stated in \cite{chetto1989some}:
\begin{align}
f_T^{EDL}(H - t) = f_t^{EDF}(t),
\end{align}
Where $H$ is the hyperperiod and $f_t$ is the availability function:
\begin{equation*}
f_T(t) = \begin{cases}
1 &\text{if processor is idle at $t$,}\\
0 &\text{else.}
\end{cases}
\end{equation*}

Consequently, EDL static deadline vector can be computed from EDF vector by simply reversing the elements.

EDL deadline vector values can be calculated according to the following formula:
\begin{align}
k_i^{EDL} = H - k_i^{EDF} - c_i.
\end{align} 

The idle function for given time instance is updated during the execution of RLP algorithm. 
Update of EDL algorithm takes into account the current red ready tasks and the current running task, if its current state is red. 
Dynamic EDL algorithm is calculated whenever a blue task is released, or a blue task is completed. 

% If there are no blue instances ready for execution, red instances are scheduled according to EDF algorithm.
% Otherwise, red instances are processed as late as possible by the EDL rule, while blue instances are executed as soon as possible in the remaining EDL idle times.
A pseudo-code for RLP algorithm is given in code listing \ref{alg:rlp}.
The tasks are stored in three lists, sorted in increasing order of deadline: 
\begin{itemize}
	\item{waiting list: list of instances waiting for their next release,}
	\item{red ready list: list of red instances in ready state,}
	\item{blue ready list: list of blue instances in ready state.}
\end{itemize}
At every time slice, the scheduler checks whether any instances changed their state from ready to waiting or vice versa and updates all three lists. 
If there is a missed instance of a task in blue ready list, task is aborted and its new instance is put into waiting list. 
Next, the waiting list is checked in order to indentify ready tasks. 
Ready tasks are put into red and blue ready list according to their state and the idle function value at current time. 
If current time belongs to an EDL idle time, the first instance in the blue ready list is selected for execution. Otherwise, the first instance from the red ready list is selected.

\begin{algorithm} % TODO staviti \small i captione
\caption{RLP scheduling algorithm.\label{alg:rlp}}
\begin{algorithmic}
\STATE initialize tasks
\FOR{task in blue\_ready}
\IF{$task_{next\_instance}$}
\STATE abort task
\ENDIF
\ENDFOR
\FOR{task in waiting\_list}
\IF{\NOT $task_{ready}$}
\STATE break
\ENDIF
\IF{$task_{state}$==RED
\AND$f_{EDL}(t)=0$}
\STATE pull task from $waiting\_list$
\STATE put task into $red\_ready$ list
\ELSE
\IF{$blue\_ready = \emptyset$}
\STATE compute EDL schedule
\ENDIF
\IF{$f\_EDL(current\_time) \neq 0$}
\STATE pull task from $waiting\_list$
\STATE put task into $blue\_ready$ list
\ENDIF
\ENDIF
\ENDFOR
\IF{$blue\_ready \neq \emptyset$ 
\AND $f\_EDL(current\_time) \neq 0$}
\FOR{task in red\_ready list}
\STATE pull task from red ready list
\STATE put task into ready list
\ENDFOR
\ENDIF
\end{algorithmic}
\end{algorithm}

An implementation of the RLP algorithm was tested on the same set of periodic tasks as in the previous sections and the resulting schedule is shown in Fig. 
\ref{rlp_schedule}.
Since there are four deadline misses, the QoS is increased to 0.73.

\begin{figure}[ht]
    \centering
    \includegraphics[width=1\textwidth]{images/skipover_RLP.pdf}
    \caption{An example of a schedule produced by RLP algorithm with skipping factor $s_i=2$.}
    \label{rlp_schedule}
\end{figure}

A comparison of QoS value depending on the utilization value for the described algorithms is shown in Fig. \ref{fm_comparison}.

\begin{figure}[ht]
    \centering
    \includegraphics[width=1\textwidth]{images/fm_comparison.pdf}
    \caption{Comparison of QoS value for RTO, BWP and RLP algorithm.}
    \label{fm_comparison}
\end{figure}

% \begin{algorithm}
% \caption{RLP scheduling algorithm.\label{alg:rlp}}
% \begin{algorithmic}
% \STATE initialize tasks 

% \IF{$running\_task$} 
% \IF{$current\_time == running\_task_{due\_date}$ 
% \AND $running\_task_{remaining\_time} > 0$}
% \STATE $running\_task_{state} = RED$ \COMMENT{running task is going to miss deadline}
% \STATE $set\_next\_instance( running\_task )$
% \STATE $waiting\_list \leftarrow running\_task$
% \ENDIF
% \ENDIF


% \WHILE{$blue\_ready\_list \neq \emptyset$}

% \STATE $task \leftarrow next(blue\_ready\_list)$
% \IF{$task_{release\_time} + task_{critical\_delay} < current\_time$} 
% \STATE break
% \ENDIF
% \STATE $task \leftarrow pull(blue\_ready\_list)$ \COMMENT{abort blue tasks}
% \STATE $set\_next\_instance( task )$
% \STATE $waiting\_list \leftarrow append(task, waiting\_list)$
% \ENDWHILE
% \COMMENT{checking waiting list in order to release tasks}
% \WHILE{$waiting\_list \neq \emptyset$}
% \STATE $task \leftarrow pull(waiting\_list)$
% \IF{$task_{release_time} > current\_time$}
% \STATE $break$
% \ENDIF 
% \IF{$task_{state}==RED$
% \AND$f_{EDL}(t)=0$}
% \STATE$task \leftarrow pull(waiting\_list)$ \COMMENT{red task release}
% \STATE$red\_ready\_list \leftarrow append(task, red\_ready\_list)$
% \ELSE
% \IF{$blue\_ready\_list \neq \emptyset$}
% \STATE$compute$ $EDL\_schedule$
% \ENDIF
% \IF{$f\_EDL(current\_time) \neq 0$}
% \STATE $task \leftarrow pull(waiting\_list)$
% \STATE $blue\_ready\_list \leftarrow append(task, red\_ready\_list)$
% \COMMENT{blue task release}
% \ENDIF
% \ENDIF
% \STATE $task_{current\_skip\_value} += 1$
% \ENDWHILE
% \IF{$blue\_ready\_list \neq \emptyset$ 
% \AND $f\_EDL(current\_time) \neq 0$}
% \WHILE{$task \leftarrow next(red\_ready\_list)$}
% \STATE $task \leftarrow pull(red\_ready\_list)$ \COMMENT{suspend red task}
% \STATE $waiting\_list \leftarrow append(task, waiting\_list)$
% \ENDWHILE
% \ENDIF
% \end{algorithmic}
% \end{algorithm}

\newpage
\subsection{Period Adaptation}
If a periodic task set experiences a permanent overload, one way of reducing the system load is to vary tasks' rates in order to keep the total load under below a desired threshold.
A method of reducing the system load by enlarging task periods is called period adaptation. 
Whenever the system cannot guarantee the acceptance of a new task, instead of rejecting it, the system can try to reduce the utilizations of other tasks by increasing their periods. In that way, the total load is decreased and the new task may be accepted.

A method for changing task periods as a function of the desired workload is the elastic framework. Each task is considered as flexible as a spring, whose utilization can be modified by changing its period within a specified range 
\cite{lee2007handbook}.

Each task $\tau_i$ is detoned by:
\begin{equation*}
\tau_i(C_i, T_{i_{min}}, T_{i_{max}}, E_i),
\end{equation*}
where $C_i$ is computation time, $T_{i_{min}}$ and $T_{i_{max}}$ are its minimum and maximum period, respectively.
Minimum period is considered as a nominal period. 
 $E_i$ is the elastic coefficient which specifies the flexibility of the task.
Corresponding to $T_{i_{min}}$ and $T_{i_{max}}$, the maximum and minimum utilization of the task set is defined in the following way:
\begin{align*}
U_{max} = \sum_{i=1}^{n}\frac{C_i}{T_{min}},\\
U_{min} = \sum_{i=1}^{n}\frac{C_i}{T_{max}}.
\end{align*}

For example, in a task set scheduled by EDF, the elastic model can be used to adapt the task periods so that the task set becomes schedulable:
\begin{equation*}
\sum_{i=1}^{n}\frac{C_i}{T_i} = U_d \leq 1,
\end{equation*}
where $U_d$ is the desired utilization factor.
This is possible as long as the desired utilization value is grater or equal to minimum utilization value.

According to \cite{lee2007handbook}, if there are no constrains on the maximum tesk periods, the utilization of each compressed task can be computed by the following expression:
\begin{equation*}
U_i = U_{i_{max}} - ( U_{max} - U_d ) \cdot \frac{E_i}{E_{tot}},
\end{equation*}
where \( E_{tot} = \sum_{i=1}^{n}E_i \).

The main advantage of this approach is that a new period configuration can be determined online based on the values of elastic coefficients, which are set beforehand based on some design criterion.
On the other hand, in the presence of period constraints, the compression algorithm becomes iterative with complexity \( O(n^2) \) \cite{lee2007handbook}.

\subsection{Service Adaptation}
Service adaptation method reduces the system load by decreasing the task computation times. 
This approach is only applicable with the tasks whose result quality is directly related to computation time.
An example of such tasks are algortihms that produce approximated results where the precision of the results depends on the number of iterations, and thus with the computation time. 
An overload condition can be handled by reducing the quality of results, aborting the remaining computation if the quality of the current result is acceptable \cite{shih1989scheduling}. 
The described concept is known in literature as imprecise computation. 



% In a system that supports imprecise computation, every task is divided into a mandatory subtask $M_i$ and an optional subtask $O_i$. 
% Mandatory subtask must be completed in order to produce a result of acceptable quality, whereas the optional subtask refines this result 
% \cite{shih1989scheduling}. 
% Both subtasks have the same arrival time and deadline as the original task, but the optional subtask is ready for execution after the mandatory task is completed. 


\chapter{Implementation Overview}
\section{Generating Test Task Sets}
For testing the performance of the considered algorithms and heuristics, it is needed to generate a large amount of syntetic test tasks.
The simulation results should not be biased by the task generation method, so selecting the approach to random generation of the task set requires special attention.
An important factor in generating syntetic task sets is the probability density function of the random variables used to generate the task set parameters \cite{bini2005measuring}.

The first step to generating the task set parameters is generating the task periods.
Treating the task periods as random variables does not reflect the characteristics of real applications, since task periods are defined by the user and enforced by the operating system \cite{bini2005measuring}.
However, for testing the scheduling heuristics without any a priori knowledge about their future applications and the characteristics of the corresponding environment, asuuming task periods as random variables with a uniform distribution is acceptable.

After selecting the task periods, it is required to generate task computation times, according to a given distribution.
The computation time $c_i$ is assumed to have a uniform distribution scaled by a factor $T_i$:
\begin{equation*}
c_i \sim \mathcal{U}[0, T_i].
\end{equation*}
This is equivalent to assuming each task utilization $u_i$ to have a uniform distribution in the interval $[0, 1]$.

For testing purposes, it is often desireable to generate the task utilization values in a way that they correspond to the given total processor utilization value.
The stated feature is accomplished by generating the individual utilization $u_i$ with an uniform distribution with the interval $[0, \overline{U}]$, subject to the constraint:
\begin{equation*}
\sum_{i=1}^{N}u_i = \overline{U}.
\end{equation*}
The utilization disparity in a task set can be expressed by a parameter called \textit{U-difference}:
\begin{equation*}
\delta = \frac{max_i\{u_i\}-min_i\{u_i\}}{\sum_{i=1}^{n}u_i}.
\end{equation*}
If $\delta = 0$, all utilization factors are the same, whereas $\delta=1$ denotes the maximum degree of difference \cite{bini2005measuring}.

To efficiently generate utilization factors for the task set with given utilization factor $\overline{u}$, where $\delta \rightarrow 1$, the \textit{Uunifast} algorithm is used.

\subsection{Uunifast Algorithm}
The UUniFast algorithm is built on the consideration that the probability density function of the sum of independent random variables is given by the convolution of ther probability dension functions \cite{bini2005measuring}.
In every algorithm iteration, a value of the sum of variables is randomly generated.
The single utilization factor $u_i$ is set equal to the difference between $\overline{u}$ and the generated value.
The complexity of the algorithm is $O(n)$.
A pseudocode describing the algorithm is shown in code listing \ref{uunifast}.
The algorithm inputs are the number of variables \texttt{n} and mean utilization factor, \texttt{mean\_u}.
\begin{algorithm}
\caption{Uunifast algorithm.\label{uunifast}}
\begin{algorithmic}
\STATE sum\_u = mean\_u
\FOR{i=1 \TO n-1}
\STATE next\_sum\_u = sum\_u * rand \^{} (1/(n-i))
\STATE vec\_u[i] = sum\_u - next\_sum\_u
\STATE sum\_u = next\_sum\_u
\ENDFOR
\STATE vec\_u[n] = sum\_u
\end{algorithmic}
\end{algorithm}
Figure \ref{uunifast:fig} illustrates the values of 10000 utilization tuples generated by the UUniFast algorithm.
The value of \texttt{n} is set to 3, while the mean utilization factor $\overline{u}$ is set to 1.
The density of the generated values is uniform in the interval $[0, 1]$.
\begin{figure}[ht]
    \centering
    \includegraphics[width=1\textwidth]{images/rsz_uunifast.png}
    \caption{Utilization tuples generated by the UUniFast algorithm.}
    \label{uunifast:fig}
\end{figure}

The generation of the test sets of periodic tasks is implemented in the \texttt{UunifastCreator} class by the \texttt{create\_test\_set()} method.
The required number of tasks and the overload factor are set through the class constructor.
First, the task utilizations are created by the \texttt{generate\_utils()} method which implements the algorithm described in code listing \ref{uunifast}.
The task periods are generated as random variables with logarithmic uniform distribution.
This is implemented in the \texttt{generate\_log\_uniform()} method.

\section{Heuristics Evolution}
The genetic programming evolution is implemented in the \texttt{GeneticAlgorithm} class.
The population size and generation number parameters are set through the class constructor.
Each individual is represented as an object of class \texttt{TreeSolution}.
This class contains two variable members: the fitness value of the individual and a pointer to the tree genotype.

\subsection{Genotype Description}
The priority functions evolved by genetic programming are represented in the form of a tree.
The tree genotypes correspond to arithmetic expressions used to calculate the priority of each job.
As described in the second chapter, a tree genotype consists of function and terminal nodes.
The terminal nodes used for assembling a priority function are listed in the following table.
\begin{table}[]
\centering
\begin{tabular}{|
>{\columncolor[HTML]{EFEFEF}}c |
>{\columncolor[HTML]{FFFFFF}}l |}
\hline
\textbf{Terminal name} & \multicolumn{1}{c|}{\cellcolor[HTML]{EFEFEF}\textbf{Definition}} \\ \hline
pt                     & nominal processing time of a job ($p\_j$)                          \\ \hline
dd                     & due date ($d\_j$)                                                  \\ \hline
w                      & weight ($w\_j$)                                                    \\ \hline
SL                     & positive slack, $SL = max\{d\_j - p\_j - time, 0\}$                \\ \hline
Nr                     & number of remaining (unscheduled) jobs                           \\ \hline
SPr                    & sum of processing times of all remaining jobs                    \\ \hline
SD                     & sum of due dates of all jobs                                     \\ \hline
\end{tabular}
\end{table}
The values of the variables in terminal node set depend not only on the parameters of the current job that is being dispatched, but also on the parameters of remaining (unscheduled) jobs.

The following table contains a list of the function nodes.
\begin{table}[H]
\centering
\label{tbl:functions}
\begin{tabular}{|
>{\columncolor[HTML]{EFEFEF}}c |
>{\columncolor[HTML]{FFFFFF}}c |}
\hline
\textbf{Function name} & \cellcolor[HTML]{EFEFEF}\textbf{Definition}                  \\ \hline
ADD, SUB, MUL, DIV     & Addition, subtraction, multiplication and protected division \\ \hline
POS                    & $POS(a) = max\{a, 0\}$                                         \\ \hline
\end{tabular}
\end{table}
Protected division node is a modified division operator which checks whether the second argument (denominator) is zero before performing division.
If the second argument is zero, it returns the value 1, regardless of the value of the first argument.

The genotype primitives are described by objects of the base class \texttt{AbstractNode}.
Child nodes of each primitive are stored as a vector of pointers to \texttt{AbstractNode} objects.
% The \texttt{children} vector is a member of the \texttt{AbstractNode} class.
Parameters and actions for specific primitives are handled by the derived classes, where every class corresponds to a function or terminal node.
Each class contains an \texttt{execute()} method which recursively invokes the same method upon every child node.
In terminal nodes, the \texttt{execute()} method returns the value of the corresponding task parameter.

Generating the tree genotypes is handled by the methods of the \texttt{TreeConstructor} class.
There are two basic approaches for generating a genotype: the \textit{full} method and the \textit{grow} method \cite{koza1992genetic}.
The \textit{full} method is used for generating full trees. 
It sets the root node as a random function or terminal node.
If root is a function node, the root's children nodes are set to random function nodes.
The procedure is the same for every node until the required depth is reached.
The nodes at final depth are set to random terminal nodes.
On the other hand, the \textit{grow} method selects any node (function or terminal) in each step.
Both approaches are implemented as methods of the \texttt{TreeConstructor} class and their prototypes are stated in code listing \ref{treeconstr}.
\begin{lstlisting}[frame=none, label={treeconstr}, caption={Functions for implementing the \textit{full} and \textit{grow} methods for creating a genotype.}, captionpos=b]
void construct_tree_full( int max_depth, AbstractNode *&root );
void construct_tree_grow( int max_depth, AbstractNode *&root );
\end{lstlisting}
The \texttt{max\_depth} argument determines the maximum depth of the tree.
Actual depth has a random value in the interval $[1, \texttt{max\_depth}]$.
Figure \ref{fullgrow} depicts the difference between the trees generated by \textit{full} and \textit{grow} methods.
\begin{figure}[ht]
    \centering
    \includegraphics[width=1\textwidth]{images/fullgrow.pdf}
    \caption{Examples of genotypes generated by the \textit{full} (left) and \textit{grow} method (right).}
    \label{fullgrow}
\end{figure}

\subsection{Operators}
The selection, crossover and mutation operators are implemented in the \texttt{SelectionOperator}, \texttt{CrossoverOperator} and \texttt{MutationOperator} classes, respectively.
Selection is performed by the \texttt{get\_members} method whose prototype is stated in code listing \ref{selection}.
\begin{lstlisting}[frame=none, label={selection}, caption={Prototype of the \texttt{get\_members} method which performs selection.}, captionpos=b]
template <typename T>
void TreeSelection<T>::get_members( std::vector<T> &population, 
	std::vector<T> &members );
\end{lstlisting}
The selected individuals are stored in the \texttt{members} vector.

Crossover operator is implemented by the \texttt{get\_children} method whose prototype is stated in code listing \ref{crossover}.
\begin{lstlisting}[frame=none, label={crossover}, caption={Prototype of the \texttt{get\_children} method which performs crossover.}, captionpos=b]
template <typename T>
void TreeCrossover<T>::get_children( std::vector<T> &parents, 
	std::vector<T> &children );
\end{lstlisting}
Crossover is performed by selecting a random subtree of every parent genotype.
This is done by invoking the \texttt{pick\_random} method implemented in \texttt{AbstractNode} class which returns a pointer to a random child node at a random depth.
The child genotypes are generated by swapping the randomly selected subtrees.
If this operation would cause the tree depth to exceed the maximum tree depth, crossover is not performed.

Mutation of a genotype is done through the \texttt{mutate\_solution} method.
Prototype of this method is stated in code listing \ref{mutation}.
\begin{lstlisting}[frame=none, label={mutation}, caption={Prototype of the \texttt{mutate\_solution} method.}, captionpos=b]
template <typename T>
void TreeMutation<T>::mutate_solution ( T &solution );
\end{lstlisting}
The first step of the \texttt{mutate\_solution} method is generating a new subtree which shall replace a randomly selected subtree of the individual.
The new subtree is generated by invoking either the \texttt{construct\_tree\_full} or the \texttt{construct\_tree\_grow} method with equal probability.
The subtree replacement is performed by the \texttt{replace\_random} method implemented in the \texttt{AbstractNode} class.
This method takes a pointer to a newly created subtree and replaces a random child at a random depth with the subtree.

\section{Heuristics Evaluation}
\subsection{Periodic Tasks Simulator}
In this work, the tasks are represented as objects of class \texttt{Task}.
The following parameters are set through the constructor:
\begin{itemize}
	\item task period,
	\item duration,
	\item relative due date,
	\item weight,
	\item skip factor,
	\item task ID.
\end{itemize}
The task priority computed by the heuristic is stored in the \texttt{priority} member variable.
Additional parameters used for tracking and managing task execution include:
\begin{itemize}
	\item instance counter,
	\item arrival time,
	\item absolute due date,
	\item state,
	\item tardiness.
\end{itemize}
When a new task is created, these parameters are initialized by the \texttt{initialize\_task()} method.
The \texttt{instance\_counter} variable is used for tracking which instance of a task is active and computing the arrival time and due date of the next instance.
Upon the appearance of a new task, its absolute due date and state parameters are updated.
The state parameter is used in skip-over task model and it is represented by a variable of enumerated type. Its initial value is set to \texttt{RED}.
The states of the following instances are set by the \texttt{update\_rb\_params()} method.
If the skipover model is not being used, the task tardiness is tracked.
The tardiness value of a task is updated upon the end of the task execution through the \texttt{update\_tardiness()} method.

\section{Support for Multicriterial Optimization and Cooperative Coevolution}
\section{Integrating Evolved Scheduling Heuristics into FreeRTOS}
\subsection{FreeRTOS Task Management}
FreeRTOS is an open-source real-time operating system designed for embedded systems. 
Its main features include portability, simplicity of the source code, and binary code compactness. 

The minimum FreeRTOS kernel code is contained in three source files \cite{brown2012architecture}. 
The code that handles task creating and scheduling is situated in the source file \verb$tasks.c$ and header file \verb$task.h$.

The tasks are managed through the Task Control Block (TCB) structure. 
A TCB corresponding to each task contains all information necessary to completely describe the task state. 
The TCB fields include task name, initial priority, unique TCB number and a pointer to the top of the task's stack. 
When a task is added to a list, it is represented by a pointer to a \verb$ListItem$ object. 
The TCB structure contains two \verb$ListItem$ objects: \verb$xStateListItem$ and \verb$xGenericListItem$.

A task in FreeRTOS can exist in one of the following states: deleted, suspended, ready, blocked and running. 
Figure \ref{freertos:state} shows a state diagram for FreeRTOS tasks. 

\begin{figure}[ht]
    \centering
    \includegraphics[width=0.70\textwidth]{images/freertos_fsm.pdf}
    \caption{State diagram of FreeRTOS tasks.}
    \label{freertos:state}
\end{figure}

Task states are tracked implicitly by placing tasks in the appropriate lists: ready list, suspended list, etc. As a task changes state, it is simply moved from one list to another 
\cite{brown2012architecture}.

A task is created by the \verb$xTaskCreate()$ function. 
The user-defined parameters required to create a task include: 
\begin{itemize}
	\item a pointer to the function that implements the task,
	\item the task name,
	\item the depth of the task's stack,
	\item the task's priority,
	\item a pointer to any parameters needed by the task function.
\end{itemize}
The \verb$xTakCreate()$ first allocates memory for the task's TCB and stack.
Next, the TCB fields are initialized with the task name, priority and stack depth from function parameters. 
Finally, a pointer to the top of the task's stack is initialized and the stack is populated with a \textit{dummy frame}. 
In that way, the task is prepared for its first context switch \cite{goyette2007analysis}.

After all the required tasks have been created, the FreeRTOS scheduler is started by a call to the 
\verb$vTaskStartScheduler()$ function. 
First, the Idle task with the lowest priority is created.
The global timer \verb$xTickCount$ is set to zero. 
The \verb$vTaskStartScheduler()$ function then passes control to the \verb$xTaskStartScheduler()$ in the Hardware Abstraction Layer (HAL), which configures the timer interrupt needed for invoking the scheduler. 
The HAL scheduler is also in charge of restoring the context of the currently selected task
\cite{goyette2007analysis}. 

Newly created tasks are placed into the ready state and added to the ready list. 
The ready list is implemented as an array of task lists:
\begin{lstlisting}[frame=none, label={lst:readylist}, caption={Ready task list}, captionpos=b]
static List_t pxReadyTasksLists[configMAX_PRIORITIES];
\end{lstlisting}
The elements of the \verb$pxReadyTasksLists$ array are lists of tasks that have the same priority, from \verb$0$ to \verb$configMAX_PRIORITIES-1$.
An example of a ready list is shown in figure \ref{freertos:ready}.
There are three priority levels in the list: task A has priority 0, no tasks have priority 1 and tasks B, C and D have priority 2. 

\begin{figure}[ht]
    \centering
    \includegraphics[width=0.70\textwidth]{images/ready_list.pdf}
    \caption{A schematic view of the FreeRTOS Ready List. Modified from \cite{brown2012architecture}.}
    \label{freertos:ready}
\end{figure}
When a new task is added to the ready list, its \verb$xStateListItem$ is inserted at the end of the 
associated priority level list.

A task in the running state is identified by the \verb$pxCurrentTCB$ variable, which is updated at every system tick interrupt. 
Every time the tick interrupt occurs, the \verb$xTaskSwitchContext()$ function is called and it selects the highest-priority ready task.
After the highest priority level is determined,
the highest-priority task is selected by the \verb$listGET_OWNER_OF_NEXT_ENTRY()$ function.
The function traverses the priority level's ready list and assigns the next ready task to the \verb$pxCurrentTCB$ variable.

The tasks enter the blocked state when they are waiting for time related or synchronization events. 
A task can be placed into the blocked state by calling the \verb$vTaskDelay()$ and \verb$vTaskDelayUntil()$ API functions. 
The \verb$vTaskDelayUntil()$ function defines the frequency at which the task is periodically executed and therefore it can be used to implement periodic tasks \cite{carraro2016implementation}.
The function sets the value of the tick at which the task will activate into the \verb$xStateListItem$ element and places it in the delayed list:
\begin{lstlisting}[frame=none, label={lst:delay}, caption={Transition to blocked state.}, captionpos=b]
listSET_LIST_ITEM_VALUE(&(pxCurrentTCB->xStateListItem), 
						xTimeToWake);
vListInsert(pxDelayedTaskList, &(pxCurrentTCB->xStateListItem));
\end{lstlisting}
The \verb$vListInsert()$ function sorts the elements by the \verb$xStateListItem$ values. 
At every increment of the tick count, it must be checked whether a task needs to be unblocked. 
This is implemented in the \verb$xTaskIncrementTick()$ function which is called from the HAL every time the timer interrupt occurs.
The nearest unblock tick value is stored in the \verb$xNextUnblockTime$ variable, and it corresponds to the unblock time of the first element of the delayed list.
When the current tick value achieves the \verb$xNextUnblockTime$ value, the first task from the delayed list is retrieved by the \verb$listGET_OWNER_OF_HEAD_ENTRY()$ function.
The task is placed in the ready list and the function return value signifies that a context switch is needed.

The suspended state is assumed when the \verb$vTaskSuspend()$ API function is called and the tasks are switched back from suspended state by \verb$vTaskResume()$ function.

\subsection{FreeRTOS Scheduler Modification}
FreeRTOS uses a static priority policy for task scheduling. 
To achieve dynamic priority assignment, the FreeRTOS task management subsystem must be modified. 
This is possible by modifying the existing FreeRTOS functions and data structures, but also adding new objects that allow dynamic task priorities to be managed. 

The general idea of implementing a dynamic scheduler is to create a new ready list which contains tasks ordered by a custom parameter defined in \verb$xTaskStateItem$ object.
In this case, \verb$xTaskStateItem$ shall contain the task priority computed by the evaluated heuristic and the list will be sorted in increasing priority value. 
The priority of the Idle task is set to some arbitrary value, significantly greater than the other tasks' priorities. 

To determine whether dynamic scheduling based on the evolved heuristic is used, a configuration variable \verb$configUSE_GP_SCHEDULER$ is added to the \verb$FreeRTOSConfig.h$ file. 

As described in the previous sections, the task parameters used for priority computation include the task period, duration, deadline and weight. 
These parameters need to be added to the TCB structure. 
Additionally, the task priority value computed by the heuristic is added, as well as the \verb$xRemainingTicks$ variable used for tracing task execution.
\begin{lstlisting}[frame=none, label={TCB}, caption={Modification of the TCB.}, captionpos=b]
typedef struct tskTaskControlBlock
{
	...
	#if( configUSE_GP_SCHEDULER == 1 )
		TickType_t xTaskPeriod;
		TickType_t xTaskDuration;
		TickType_t xDeadline;
		TickType_t xRemainingTicks;
		double xTaskWeight;
		double xPriorityValue;
	#endif
} tskTCB;
\end{lstlisting}
The additional parameters are set through the \verb$xTaskPeriodicCreate()$ function, which is a modified version of the standard \verb$xTaskCreate()$.
\begin{lstlisting}[frame=none, label={periodicCreate}, caption={The \texttt{xTaskPeriodicCreate()} function prototype.}, captionpos=b]
BaseType_t xTaskPeriodicCreate(	TaskFunction_t pxTaskCode,
						const char * const pcName,		
						const configSTACK_DEPTH_TYPE usStackDepth,
						void * const pvParameters,
						UBaseType_t uxPriority,
						TaskHandle_t * const pxCreatedTask,
						TickType_t period,
						TickType_t duration,
						uint32_t weight ) PRIVILEGED_FUNCTION
\end{lstlisting}
The user-defined parameters are added to the new task's TCB structure. 
\begin{lstlisting}[frame=none, label={TCB_params}, caption={Adding the user-defined task parameters to the TCB structure.}, captionpos=b]
pxNewTCB->xTaskPeriod = period;
pxNewTCB->xTaskDuration = duration;
pxNewTCB->uTaskWeight = weight;
pxNewTCB->xDeadline= period;
pxNewTCB->xRemainingTicks = duration;
\end{lstlisting}
After the task priority value is computed, this value is assigned to the task's \\\verb$xStateListItem$ element. 
\begin{lstlisting}[frame=none, label={init_priority}, caption={Assigning the task priority to the \texttt{xStateListItem} element.}, captionpos=b]
vTaskComputePriority( pxNewTCB );
listSET_LIST_ITEM_VALUE( &((pxNewTCB)->xStateListItem), 
						(pxNewTCB)->xPriorityValue );
\end{lstlisting}
Next, the new ready tasks list must be declared.
\begin{lstlisting}[frame=none, label={ready_list}, caption={Declaration of the new ready tasks list.}, captionpos=b]
#if( configUSE_GP_SCHEDULER == 1 )
	PRIVILEGED_DATA static List_t xReadyTasksListGP;
#endif /* configUSE_GP_SCHEDULER */
\end{lstlisting}
Before adding a task to the ready list, its priority value is computed. 
Since the Idle task is added to the ready list in the same way as other tasks, it has to be distinguished in some manner. 
While creating the Idle task, its duration is set to 0.
Therefore, before assigning a priority value to a task, its duration value is checked. 
An example of the priority computation according to an arbitrary heuristic is given in the following code listing. 
In this example, priority is computed according to the expression:
\begin{align*}
p_i = d_i + c_i + max\{ d_i - c_i - current\_time, 0 \}.
\end{align*}
The Idle task priority is set to 1000.
\begin{lstlisting}[frame=none, label={ready_list}, caption={Macro function for priority computation.}, captionpos=b]
#define vTaskComputePriority( pxTCB )
{
	pxTCB->xPriorityValue = ( pxTCB->xTaskDuration != 0 ) ? 
		( pxTCB->xDeadline + pxTCB->xTaskDuration + 
		max( pxTCB->xDeadline - pxTCB->xTaskDuration, 0 )) : 1000;
}	
\end{lstlisting}
The \verb$prvAddTaskToReadyList()$ function is modified in a way that the \verb$vListInsert()$ method is used instead of \verb$vListInsertEnd()$. 
The elements are sorted by the \\\verb$xStateListItem$ element, so the lowest priority tasks are placed at the beginning of the list. 
\begin{lstlisting}[frame=none, label={ready_add}, caption={Adding a new task to the ready list.}, captionpos=b]
#if ( configUSE_GP_SCHEDULER == 1 )
	#define prvAddTaskToReadyList( pxTCB )
	{
		vTaskComputePriority( pxTCB );
		listSET_LIST_ITEM_VALUE( &((pxTCB)->xStateListItem), 
								(pxTCB)->xPriorityValue );
		vListInsert(&(xReadyTasksListGP), 
					&((pxTCB)->xStateListItem) );
	}																								
#else 																								
	#define prvAddTaskToReadyList( pxTCB )
	{
		vListInsertEnd(&(pxReadyTasksLists[(pxTCB)->uxPriority] ), 
						&((pxTCB)->xStateListItem));
	}
#endif
\end{lstlisting}
The \verb$prvInitialiseTaskLists()$ must be called in order to initialise the elements of the ready tasks list.
\begin{lstlisting}[frame=none, label={ready_list}, caption={Declaration of the new ready tasks list.}, captionpos=b]
#if( configUSE_GP_SCHEDULER == 1 )
	vListInitialise( &(xReadyTasksListGP) );
#endif /* configUSE_GP_SCHEDULER */
\end{lstlisting}
When a new task is added to the ready list while the scheduler is not running, its priority value is compared to the priority of the running task. 
If the new task has a higher priority, it is set as the currently running task.
Therefore, a modification in the \verb$prvAddNewTaskToReadyList()$ function is required in order to include the comparison between computed priority values. 
\begin{lstlisting}[frame=none, label={newtask}, caption={Modification of the \texttt{prvAddNewTaskToReadyList()} function.}, captionpos=b]
#if( configUSE_GP_SCHEDULER == 1 )
	vListInitialise( &(xReadyTasksListGP) );
#endif /* configUSE_GP_SCHEDULER */
\end{lstlisting}
The \verb$vTaskStartScheduler()$ is also modified in order to manage the creation of the Idle task.
\begin{lstlisting}[frame=none, label={idle}, caption={Creation of the Idle task.}, captionpos=b]
#if( configUSE_GP_SCHEDULER == 1 )
{
	xReturn = xTaskPeriodicCreate( prvIdleTask,
					configIDLE_TASK_NAME,
					configMINIMAL_STACK_SIZE,
					( void* ) NULL,
					( tskIDLE_PRIORITY | portPRIVILEGE_BIT ),
					&xIdleTaskHandle,
					0,
					0,
					0 );
}
#else 
	...
\end{lstlisting}
The \verb$vTaskSwitchContext()$ function is modified in a way that it selects the first element of the ready list as the currently running task, instead of calling the \\\verb$taskSELECT_HIGHEST_PRIORITY_TASK()$ function.
\begin{lstlisting}[frame=none, label={switchcontext}, caption={\texttt{vTaskSwitchContext()} modification.}, captionpos=b]
#if( configUSE_GP_SCHEDULER == 1 )
{
	pxCurrentTCB =
		(TCB_t *)listGET_OWNER_OF_HEAD_ENTRY(&(xReadyTasksListGP));
}
#else 
{
	taskSELECT_HIGHEST_PRIORITY_TASK();
}
#endif
\end{lstlisting} 
Since absolute deadline value is required for priority computation, this parameter must be updated at each end of a task instance.
If the \verb$xRemainingTicks$ variable reached zero before the task is switched out, \verb$xDeadline$ parameter is updated.
\begin{lstlisting}[frame=none, label={taskfinish}, caption={\texttt{Updating the \texttt{xDeadline} parameter at the end of task instance}.}, captionpos=b]
#if( configUSE_GP_SCHEDULER == 1 )
{
	if( pxCurrentTCB->xRemainingTicks == 0 ) 
	{
		pxCurrentTCB->xRemainingTicks = 
			pxCurrentTCB->xTaskDuration;
		pxCurrentTCB->xDueDate += pxCurrentTCB->xTaskPeriod;
	}
}
#endif
\end{lstlisting}
Finally, the last modifications are applied to the \verb$xTaskIncrementTick()$ function. 
The previous subsection described how a task is unblocked and placed into the ready list.
When a task is blocked, its \verb$xStateListItem$ element refers to the unblock tick value. 
Therefore, before placing an unblocked task into the ready list, its \verb$xStateListItem$ element must be assigned to the priority value.
Priority of the unblocked task is compared to the running task's priority in order to determine whether a context switch is needed.
Hence, the priority of the task in running state is also updated.
\begin{lstlisting}[frame=none, label={switchcontext}, caption={Managing an unblocked task.}, captionpos=b]
#if( configUSE_GP_SCHEDULER == 1 )
{
	vTaskComputePriority( pxTCB );
	listSET_LIST_ITEM_VALUE( &((pxTCB)->xStateListItem), 
							pxTCB->xPriorityValue );
	vaskComputePriority( pxCurrentTCB );
}
#endif
\end{lstlisting}

\chapter{Results}
\label{results}
\section{Comparison of Formal Methods and Genetic Programming}
This chapter shall present the performance of a simulated real-time system where the scheduling heuristic is generated by genetic programming.
The results shall be compared to the RLP scheduling algorithm, as it is the most sophisticated of all formal algorithms presented in this work.
The comparison is made with respect to various parameters: mean task skip factor, the \textit{Gini} coefficient, QoS and wasted CPU time.

In the test examples, the NSGA-II variant of the genetic algorithm was used.
The parameters of the genetic algorithm are listed in the following table.
\begin{table}[H]
\centering
\begin{tabular}{|
>{\columncolor[HTML]{EFEFEF}}l |
>{\columncolor[HTML]{FFFFFF}}c |}
\hline
Population size    & 20 \\ \hline
Generation number  & 50 \\ \hline
Maximum tree depth & 5  \\ \hline
\end{tabular}
\end{table}
The heuristic was trained on task sets with six different utilization factors: 
\begin{equation*}
\{ 0.9, 1, 1.1, 1.2, 1.3, 1.4 \} \, .
\end{equation*}
$10$ different task sets were created for each utilization factor, which gives the total amount of $60$ training task sets.
For testing the heuristic, the utilization factor varied from $0.9$ to $1.6$ with $0.05$ increment.
For evaluation, $100$ different task sets were generated per utilization factor, i.e., $10$ times more than for training.
The task sets for both training and testing were generated by the \textit{UUniFast} algorithm.

% Fig. \ref{heur_v_rlp} shows the test results for an evolved heuristic defined by the following expression:
% \begin{equation*}
% p_i = c_i \cdot (w_i + d_i) \, .
% \end{equation*}

Fig. \ref{heur_v_rlp} shows a comparison of the results achieved by the evolved heuristic and the results of RLP algorithm with respect to QoS metric.
Moreover, the results of three approaches for heuristic evolution are compared:
single-objective optimization, multi-objective optimization and cooperative coevolution.
QoS values for all the task sets are shown by the scatter plot, while the line graph shows a polynomial regression model of the scattered data.
% The evolved heuristic is compared to the RLP algorithm with respect to QoS metric.
% QoS values for all the task sets are shown by the scatter plot.
% The line graph shows the eighth degree polynomial regression model of the scattered data.
\begin{figure}[ht]
    \centering
    \includegraphics[width=1\textwidth]{images/heur3_v_rlp.pdf}
    \caption{A comparison of the RLP algorithm and the evolved heuristic with respect to the QoS metric.}
    \label{heur_v_rlp}
\end{figure}
For underload coditions, the best result is achieved by the RLP algorithm.
However, for task sets with utilization factors greater than $1.1$, the heuristic evolved by multi-objective optimization yields significantly better results in comparison with RLP algorithm and heuristic evolved by single-objective coevolution.
For utilization factor $1.6$, the QoS achieved by multi-objective optimization surpasses RLP and single-objective optimization by 30\%.
QoS achieved by the coevolution approach is degraded in comparison with the QoS achieved by multi-objective optimization with a single heuristic.
A potential benefit of using cooperative coevolution is described with Fig. \ref{heur_v_rlp_wCPU}.
The described approaches are compared with respect to the wasted CPU time metric.
The value of wasted CPU time for each set is limited to the interval $[0, 1]$ by dividing it with the hyperperiod of the task set.
As the results are significantly dispersed, the mean value of wasted CPU time for each utilization factor with $0.05$ resolution is shown instead of polynomial regression model.
The heuristics evolved by coevolution approach produced schedules with less wasted CPU time in overload conditions.
\begin{figure}[ht]
    \centering
    \includegraphics[width=1\textwidth]{images/mean_wCPU.pdf}
    \caption{A comparison of the RLP algorithm and the evolved heuristic with respect to the wasted CPU time metric.}
    \label{heur_v_rlp_wCPU}
\end{figure}

In the continuation, the performance of the evolved heuristic shall be presented through two examples of task sets.
The utilization of both task sets equals 1.2 and the minimal skip factor for the RLP algorithm is set to 2.
% The considered heuristic is described by the following expression:
% \begin{equation*}
% c_i \cdot \frac{S_d}{\frac{S_d}{c_i}} = c_i^2.
% \end{equation*}

In the first example, the schedule produced by the heuristic is compared to the RLP schedule from Section \ref{skip_algs} shown in Fig. \ref{rlp_schedule}.
QoS metric achieved by the RLP algorithm for this task set equals $0.73$, with a total of $9$ time units of wasted CPU time.
The schedule for this task set provided by the evolved heuristic is shown in Fig. \ref{rlp_heur_comp1}.
\begin{figure}[ht]
    \centering
    \includegraphics[width=1\textwidth]{images/heur_vs_RLP_example1.pdf}
    \caption{A schedule produced by the evolved heuristic for task set 1.}
    \label{rlp_heur_comp1}
\end{figure}
This schedule results in two additional successfully completed instances, increasing the QoS metric to $0.87$. 
There is only one aborted task instance and the amount of wasted CPU time equals $4$.

The second task set is defined by the following parameters:
\begin{table}[H]
\begin{center}
\begin{tabular}{|
>{\columncolor[HTML]{FFFFFF}}c |c|c|}
\hline
   & \cellcolor[HTML]{FFFFFF}\textbf{$T_i$} & \cellcolor[HTML]{FFFFFF}\textbf{$c_i$} \\ \hline
\textbf{$\tau_1$} & 15                         & 6                          \\ \hline
\textbf{$\tau_2$} & 12                          & 4                          \\ \hline
\textbf{$\tau_3$} & 6                          & 3                          \\ \hline
\end{tabular}
\end{center}
\end{table}

Fig. \ref{rlp_example_2} shows the schedule produced by the RLP algorithm.
\begin{figure}[ht]
    \centering
    \includegraphics[width=1\textwidth]{images/skipover_RLP_3.pdf}
    \caption{A schedule produced by RLP algorithm for task set 2.}
    \label{rlp_example_2}
\end{figure}
In this example, the amount of skipped or aborted tasks equals 8 instances of total 19.
The aborted instances were executing for a total of $19$ time units.

The schedule provided by the evolved heuristic is shown in Fig. \ref{heur_example_2}.
\begin{figure}[ht]
    \centering
    \includegraphics[width=1\textwidth]{images/heur_vs_RLP_example2.pdf}
    \caption{A schedule produced by the evolved heuristic for task set 2.}
    \label{heur_example_2}
\end{figure}
The number of successfully completed tasks is increased to 13.
Regarding the QoS metric and the wasted CPU time, the heuristic yields better results.
However, the schedule results in two consecutive skipped instances of the same task.
Consequently, there is a significant difference between mean skip factors of the tasks in the task set.
Therefore, the heuristic is not applicable to this task set.

% The parameters of the first task set are defined as follows:
% \begin{table}[htbp]
% \begin{center}
% \begin{tabular}{|
% >{\columncolor[HTML]{FFFFFF}}c |c|c|}
% \hline
%    & \cellcolor[HTML]{FFFFFF}\textbf{$T_i$} & \cellcolor[HTML]{FFFFFF}\textbf{$c_i$} \\ \hline
% \textbf{$\tau_1$} & 10                         & 2                          \\ \hline
% \textbf{$\tau_2$} & 6                          & 2                          \\ \hline
% \textbf{$\tau_3$} & 3                          & 2                          \\ \hline
% \end{tabular}
% \end{center}
% \end{table}
% Fig. \ref{rlp_1} shows a schedule produced by RLP algorithm. 
% \begin{figure}[ht]
%     \centering
%     \includegraphics[width=1\textwidth]{images/res_skipover_RLP.pdf}
%     \caption{A schedule produced by RLP algorithm for task set 1.}
%     \label{rlp_1}
% \end{figure}
% The skip factors for tasks $\tau_1$, $\tau_2$ and $\tau_3$ equal $4$, $6$, and $2$, respectively.
% The mean skip factor for all tasks equals $4.17$.
% Since there are four aborted or skipped instances of task $\tau_3$, the QoS parameter equals $0.78$.
% The two aborted instances of task $\tau_3$ at time instants $t=18$ and $t=24$ result in $2$ units of wasted CPU time.

% Fig. \ref{heur_1} shows the schedule for the first task set produced by an evolved heuristic.
% This task set was used as a training set for heuristic evolution.
% The heuristic is represented by the following expression:
% \begin{equation*}
% p_i = N_r \cdot c_i,
% \end{equation*}
% where $p_i$ is the priority of the current task, $N_r$ is the number of remaining (unscheduled) jobs and $c_i$ is the processing time. 
% The relating schedule is shown in figure \ref{heur_1}.
% \begin{figure}[ht]
%     \centering
%     \includegraphics[width=1\textwidth]{images/skipover_heur.pdf}
%     \caption{A schedule produced by evolved heuristic for task set 1.}
%     \label{heur_1}
% \end{figure}
% The number of successfully completed tasks is increased to 15, improving the QoS value to $0.83$.
% Since there are no aborted task instances, there is no wasted CPU time.
% The mean skip factor of task $\tau_3$ now equals $3.75$, thus the overall mean skip factor is increased to $4.25$.

% The second task set parameters are listed in the following table:
% \begin{table}[H]
% \begin{center}
% \begin{tabular}{|
% >{\columncolor[HTML]{FFFFFF}}c |c|c|}
% \hline
%    & \cellcolor[HTML]{FFFFFF}\textbf{$T_i$} & \cellcolor[HTML]{FFFFFF}\textbf{$c_i$} \\ \hline
% \textbf{$\tau_1$} & 10                         & 7                          \\ \hline
% \textbf{$\tau_2$} & 6                          & 3                          \\ \hline
% \end{tabular}
% \end{center}
% \end{table}
% Fig.s \ref{rlp_2} and \ref{heur_2} show the schedules produced by RLP algorithm and evolved heuristics.
% \begin{figure}[ht]
%     \centering
%     \includegraphics[width=1\textwidth]{images/skipover_RLP_2.pdf}
%     \caption{A schedule produced by RLP algorithm for task set 2.}
%     \label{rlp_2}
% \end{figure}
% \begin{figure}[ht]
%     \centering
%     \includegraphics[width=1\textwidth]{images/skipover_heur_2.pdf}
%     \caption{A schedule produced by evolved heuristic for task set 2.}
%     \label{heur_2}
% \end{figure}

% Both schedules yield the same amount of aborted or skipped instances.
% Regarding the QoS value only, the schedules are equal.
% The mean skip factor of the heuristic schedule equals $2.83$, which is slightly greater than the mean skip factor of the RLP schedule ($2.33$). 
% However, the latter schedule results in two consecutive skipped instances of the same task.
% Consequently, there is a significant difference between mean skip factors of the two tasks.
% Therefore, the heuristic is not applicable to this task set.

% The parameters of the second task set are given in the following table.
% \begin{table}[H]
% \begin{center}
% \begin{tabular}{|
% >{\columncolor[HTML]{FFFFFF}}c |c|c|}
% \hline
%    & \cellcolor[HTML]{FFFFFF}\textbf{$T_i$} & \cellcolor[HTML]{FFFFFF}\textbf{$c_i$} \\ \hline
% \textbf{$\tau_1$} & 15                         & 6                          \\ \hline
% \textbf{$\tau_2$} & 12                          & 4                          \\ \hline
% \textbf{$\tau_3$} & 6                          & 3                          \\ \hline
% \end{tabular}
% \end{center}
% \end{table}

% The schedules produced by the RLP algorithm and the evolved heuristic are shown in Fig.s \ref{rlp_3} and \ref{heur_3}.
% \begin{figure}[ht]
%     \centering
%     \includegraphics[width=1\textwidth]{images/skipover_RLP_3.pdf}
%     \caption{A schedule produced by RLP algorithm for task set 3.}
%     \label{rlp_3}
% \end{figure}
% \begin{figure}[ht]
%     \centering
%     \includegraphics[width=1\textwidth]{images/skipover_heur_3.pdf}
%     \caption{A schedule produced by evolved heuristic for task set 3.}
%     \label{heur_3}
% \end{figure}
% The heuristic surpasses the RLP algorithm according to the QoS and wasted CPU time measures.
% The problem with skipping consecutive time instances is more significant in this case, as it occurs with two tasks, $\tau_2$ and $\tau_3$.

The results of testing the RLP algorithm and the evolved heuristic through the two described task sets are summarized in Fig. \ref{cmp2} and Fig. \ref{cmp1}.
The characteristics of the algorithms are compared with respect to two design criterions. 
In Fig. \ref{cmp1}, the algorithms are compared with regard to QoS metric and the amount of wasted CPU time. The best result according to both criterions was achieved by the evolved heuristic. 
\begin{figure}[H]
    \centering
    \includegraphics[width=1\textwidth]{images/Qos_wCPU.pdf}
    \caption{A comparison of RLP algorithm and the evolved heuristic considering the QoS metric and the amount of wasted CPU time.}
    \label{cmp1}
\end{figure}

Fig. \ref{cmp2} shows the comparison considering the mean skip factor value and the \textit{Gini} coefficient. 
Although the heuristic yields a greater value of mean skip factor in both examples, its result regarding the \textit{Gini} coefficient is significantly worse.
The disparity of \textit{Gini} coefficients occurs because the RLP algorithm defines a maximum frequency of skipped or aborted task instances.
It ensures that a predefined number of instances will be completed between the skipped instances, regardless of the task parameters.
On the other hand, the heuristic allows skipping consecutive task instances.
Moreover, the behavior of the heuristic is conditioned by a certain task parameter.
This is emphasized in the described examples, where the heuristic preferred skipping the task with the longest computation time.
% In the task set which was used as a train set for the heuristic (first subplot), the best results regarding both the skip factor and \textit{Gini} coefficient were achieved by the evolved heuristic.
% In the remaining task sets (middle and right subplot), the best results regarding the mean skip factor were accomplished by the evolved heuristic, but the corresponding value of \textit{Gini} coefficient is the greatest. 
\begin{figure}[H]
    \centering
    \includegraphics[width=1\textwidth]{images/mean_s_stddev.pdf}
    \caption{A comparison of RLP algorithm and the evolved heuristic considering the mean skip factor value and the \textit{Gini} coefficient.}
    \label{cmp2}
\end{figure}

\section{Comparison of Implemented Framework and ECF Framework}
The Evolutionary Computation Framework\footnote{http://ecf.zemris.fer.hr} was used as a referent framework for validating the results of the implemented framework for genetic programming.
The only mandatory input parameter to the ECF framework is the genotype.
It is defined by listing the function and terminal nodes in the configuration file.
Additional parameters stated in the configuration file include the minimum and maximum tree depth, the maximum number of generations and the population size.

To use the ECF for the problem considered in this work, user-defined function and terminal nodes must be defined.
The custom nodes are implemented in separate classes, which inherit the \texttt{Tree::Primitives::Primitive} class.
The classes must implement the \texttt{execute} method:
\begin{lstlisting}[frame=none, label={exec}, caption={A prototype of the \texttt{execute} method.}, captionpos=b]
    void execute(void* result, Tree& tree);
\end{lstlisting}
This method is equivalent to the \texttt{execute} method described in Section \ref{genotype_description}.
In this case, the \texttt{execute} method is used for computing the task priority.
The context needed for computing the priority is passed to the function by the \texttt{result} parameter.
The defined function and terminal nodes are added to the \texttt{Tree} object  describing a genotype through methods \texttt{addFunction} and \texttt{addTerminal}.
The \texttt{addGenotype} method is used for adding the genotype to a \texttt{State} object.
A \texttt{State} object describes a context which contains the genotypes and the algorithm and takes care of the population size, termination condition etc.
In order to use the custom nodes, they must be stated in the configuration file as the \texttt{functionset} and \texttt{terminalset} parameters.

Evaluation of the genotypes is implemented by a user-defined class which inherits the \texttt{EvaluateOp} class.
In this case, the \texttt{TaskEvalOp} class is defined.
The \texttt{evaluate} method invokes the periodic tasks simulator described in Section
\ref{evaluation}.
The simulator is implemented as a template class, so no adjustments were required for using a genotype represented by a different object.
Training is performed on 60 different task sets with utilization factors from the set:
\begin{equation*}
\{ 0.90, 1, 1.1, 1.2, 1.3, 1.4 \} \, .
\end{equation*}
The mean skip factor of all task sets is set as a fitness value.

Upon the end of the evolution process, ECF creates a file which contains the best individual's genotype in suffix notation.
An example of a genotype evolved by ECF is shown in Fig. \ref{ECF_genotype}.

The evolved genotype was tested on a set of $1500$ different task sets, with utilization factors from the interval $[0.90, 1.60]$.
Fig. \ref{ecf_v_heur} shows the results of testing the evolved genotype, compared to the results achieved by the framework implemented in this work.
The parameters of both frameworks are equivalent and the task sets for testing were generated in the same way.
The presented results of both frameworks were achieved by single-objective optimization, where the mean skip factor is used as the fitness value.
\begin{figure}[H]
    \centering
    \includegraphics[width=0.5\textwidth]{images/ECF_genotype.pdf}
    \caption{An example of a genotype evolved by the ECF framework.}
    \label{ECF_genotype}
\end{figure}
\begin{figure}[H]
    \centering
    \includegraphics[width=1\textwidth]{images/heur_v_ECF.pdf}
    \caption{Results of testing the ECF framework compared to the results achieved by the implemented framework.}
    \label{ecf_v_heur}
\end{figure}

\section{Testing the FreeRTOS Modification}
Modification of the FreeRTOS kernel was tested on a FreeRTOS simulator for Linux systems which uses POSIX threads.
The test application creates multiple tasks with diverse period, duration and weight parameters. 
A code example for a periodic task is given in the following listing.
CPU utilization is simulated by the \verb$count$ variable whose inital value corresponds to the task processing time. 
The variable is decremented at every system tick and when it reaches zero, the \verb$vTaskDealyUntil()$ function is called. 
\begin{lstlisting}[frame=none, label={periodic_task_freertos}, caption={Task function simulating a periodic task.}, captionpos=b]
void TSK_A( void *pvParameters ) {
    TickType_t xLastWakeTimeA;
    const TickType_t xFrequency = PERIOD_A;
    TickType_t count = DURATION_A;

    TickType_t xNextTime;
    TickType_t xTime;
    xLastWakeTimeA = 0;
    
    for(;;) {
        xTime= xTaskGetTickCount();
        //While loop that simulates task duration
        while(count != 0) {
            if((xNextTime = xTaskGetTickCount()) > xTime) {
                count--;
                xTime = xNextTime;
            }
        }
        count = DURATION_A;
        vTaskDelayUntil(&xLastWakeTimeA, xFrequency);
    }
}
\end{lstlisting}
Events of interest - tick increments and context switches are monitored by using trace macros. 
The following trace macros were implemented:
\begin{itemize}
	\item \verb$traceTASK_INCREMENT_TICK( xTickCount )$,
	\item \verb$traceTASK_SWITCHED_IN()$,
	\item \verb$traceTASK_SWITCHED_OUT()$.
\end{itemize}
The \verb$traceTASK_INCREMENT_TICK()$ macro is called during the tick interrupt.
During a context switch, the \verb$traceTASK_SWITCHED_IN()$ macro contains the handle of the task about to enter the running state, while the \verb$traceTASK_SWITCHED_OUT()$ contains the handle of the task about to leave the running state \cite{freertosref}. 

For tracing the total tardiness in overload conditions, a global variable \verb$xTardiness$ is used.
The tardiness value is updated every time a task execution is finished.
\begin{lstlisting}[frame=none, label={tardiness}, caption={Updating the \texttt{xTardiness} variable.}, captionpos=b]
#if( configTRACE_TARDINESS == 1 )
{
	if( pxCurrentTCB->xDeadline < xTickCount )
	{
		xTardiness += ( xTickCount - pxCurrentTCB->xDeadline ) 
			* pxCurrentTCB->xTaskWeight;
	}
}
#endif
\end{lstlisting}

The scheduler was tested on a single task set under overload condition. 
The results will be presented through three examples with different priority calculation heuristics.
The considered task set is given in the following table.
\begin{table}[H]
\begin{center}
\begin{tabular}{|
>{\columncolor[HTML]{FFFFFF}}c |c|c|c|}
\hline
   & \cellcolor[HTML]{FFFFFF}\textbf{$T_i$} & \cellcolor[HTML]{FFFFFF}\textbf{$c_i$} & \cellcolor[HTML]{FFFFFF}\textbf{$w_i$} \\ \hline
\textbf{$\tau_1$} & 8                         & 6                      & 0.5   \\ \hline
\textbf{$\tau_2$} & 5                         & 2                      & 1     \\ \hline
\end{tabular}
\end{center}
\end{table}
The total utilization factor for the given set equals:
\begin{equation*}
\sum_{i=1}^{N}\frac{C_i}{T_i} = \frac{6}{8} + \frac{2}{5} = 1.15 \, .
\end{equation*}

The schedule for the following task set provided by the FreeRTOS default scheduler is shown in figure \ref{freertos_def}.
\begin{figure}[ht]
    \centering
    \includegraphics[width=1\textwidth]{images/freertos_default_1.pdf}
    \caption{A schedule produced by the default FreeRTOS scheduler.}
    \label{freertos_def}
\end{figure}
Both tasks are assigned equal priorities. 
Task $\tau_1$ is the first task to be executed, since it was last added to the ready list. 
The instances of task $\tau_1$ start executing every time task $\tau_1$ is unblocked: $t=8$, $t=16$, $t=24$, $t=32$.
Task $\tau_2$ instances are executed while task $\tau_1$ is in the blocked state, since the activation of task $\tau_2$ is ignored while task $\tau_1$ is executing.
Therefore, every instance of task $\tau_2$ completes after its deadline. 
The total tardiness of task $\tau_2$ equals 44.

If task weight values are taken into account, the task priorities need to be set in a way that a task with a higher weight value has a higher priority.
In this example, task weight values are scaled by 2 in order to get integer priority values.
Fig. \ref{freertos_def_2} shows a schedule produced with task $\tau_2$ priority set to 2 and task $\tau_1$ priority set to 1. 
\begin{figure}[ht]
    \centering
    \includegraphics[width=1\textwidth]{images/freertos_default.pdf}
    \caption{A schedule produced by the default FreeRTOS scheduler with task weights taken into account.}
    \label{freertos_def_2}
\end{figure}
In this example, task $\tau_2$ instances execute as soon as possible because it has a higher priority.
Task $\tau_1$ executes after task $\tau_2$ is switched to blocked state.
Upon the activation of task $\tau_2$, a context switch is performed, giving advantage to task $\tau_2$.
The total weighted tardiness of task $\tau_1$ instances equals 10.

The results of adding priority function to the scheduler will be presented through three examples.
In the first example, the task priority corresponds to the task deadline.
The result is shown in figure \ref{freertos_1}. 
\begin{figure}[ht]
    \centering
    \includegraphics[width=1\textwidth]{images/freertos_edf.pdf}
    \caption{A schedule produced by a modification of FreeRTOS scheduler, heuristic $p_i = d_i$.}
    \label{freertos_edf}
\end{figure}
Higher priority is given to the task instances with earliest deadline.
This heuristic is equivalent to the EDF algorithm.
Total tardiness of task $\tau_1$ is 8, while the $\tau_2$ tardiness equals 6.
The total weighted tardiness for this example equals 10.

In the second example, a heuristic evolved by genetic programming is used as the priority function.
The task's processing time is added to the priority function: $p_i = d_i + c_i$.
The produced schedule is depicted in Fig. \ref{freertos_1}.
\begin{figure}[ht]
    \centering
    \includegraphics[width=1\textwidth]{images/freertos_heur1.pdf}
    \caption{A schedule produced by a modification of FreeRTOS scheduler, heuristic $p_i = d_i + c_i$.}
    \label{freertos_1}
\end{figure}
Priority ties are broken in favor of the task that is already executing.
An example of such situation is time instant $t=15$ where both task $\tau_1$ and $\tau_2$ have the same priority value, $22$.
As total weighted tardiness value is used as a penalty function for heuristics evolution, the produced schedule results in no late instances of the task with greater weight value.
The total tardiness of task $\tau_1$ equals 14, so the total weighed tardiness for this example is 7.

In the last example, a more complex heuristic is used as the priority function.
The task priorities are computed according to the expression:
\begin{align*}
p_i = d_i - SL \cdot \frac{c_i}{w_i} \, .
\end{align*}
Task priority is proportional to due date, but it takes the positive slack and weight value into account.
Advantage is given to the tasks with earliest due date and gratest positive slack value.
The result is shown in Fig. \ref{freertos_2}.
\begin{figure}[ht]
    \centering
    \includegraphics[width=1\textwidth]{images/freertos_heur2.pdf}
    \caption{A schedule produced by a modification of FreeRTOS scheduler, heuristic $p_i = d_i - SL \cdot c_i / w_i$.}
    \label{freertos_2}
\end{figure}
Again, the task of greater weight value has no late instances.
The total tardiness of task $\tau_1$ equals $16$, thus the total weighted tardiness for this example equals $8$.
\chapter{Conclusion}
This thesis presented a method for optimizing the skip factor of skippable periodic tasks.
The considered task model is typical for multimedia applications, where a dynamic scheduler is required.
Furthermore, the scheduler must be able to handle overload conditions and provide frequent schedule modifications.
Formal algorithms for scheduling skippable tasks (i.e. RLP and RLP/T) bring a significant overhead when applied to large amount of tasks, due to their computational complexity \cite{onqos}.
For that reason, heuristic scheduling is more acceptable in real-time systems with timeliness constraints due to its simplicity and ability to build a schedule with no overhead.
Genetic programming is suitable for evolving a heuristic which represents a priority function used for building a schedule.

Schedules built by evolved heuristics were compared to the RLP schedules with respect to QoS measure.
In overload conditions, the results of the evolved heuristic surpass the results achieved by RLP algorithm.
For utilization factor of $1.6$, the schedule built by the heuristic achieved a $25\%$ greater QoS value.

However, heuristic scheduling does not guarantee a minimum number of completed task instances between skipped ones.
In other words, skipping consecutive task instances is allowed, which can totally degrade system performance.
Therefore, using evolved heuristics for scheduling is applicable only in situations where task parameters are known beforehand.

\newcommand{\namesigdate}[2][5cm]{%
  \begin{tabular}{@{}p{#1}@{}}
    #2 \\[2\normalbaselineskip] \hrule \\[15pt]
  \end{tabular}}

\vspace*{\fill} \noindent \hfill \namesigdate{Karla Salamun}

\bibliography{literatura}
\bibliographystyle{fer}

\begin{abstract}
This thesis presents the possibilities of using machine learning for task scheduling in overloaded real-time systems.
The existing formal algorithms for scheduling in overloaded real-time systems were investigated, as well as the methods for reducing system load in transient and permanent overload conditions.
A framework for evolving scheduling heuristics using genetic programming was implemented, with support for multi-objective optimization (NSGA-II algorithm) and cooperative coevolution.
The performance of the evolved heuristics is measured by a simulator of periodic tasks execution with preemptive scheduling.
The interface for generating task sets for training and testing is implemented using the UUniFast algorithm.
The results of testing the considered technique are compared to the results of the \textit{Red Tasks as Late as Possible} (RLP) algorithm with respect to Quality of Service (QoS) metric.
Applying the evolved heuristic to a large number of synthetically generated task sets increased the QoS achieved in overload conditions up to 50\% in comparison with RLP approach.
An existing framework for evolutionary computation is applied to the task scheduling problem in order to validate the results achieved by the implemented framework.
The FreeRTOS operating system was modified to use the evolved heuristics for task scheduling.

\keywords{genetic programming, task scheduling, real-time systems.}
\end{abstract}

\newpage
% TODO: Navedite naslov na hrvatskom jeziku.
\hrtitle{Primjena genetskog programiranja za raspoređivanje zadataka u sustavima za rad u stvarnom vremenu}
\begin{sazetak}
U ovom radu prikazane su mogućnosti primjene strojnog učenja za raspoređivanje zadataka u preopterećenim sustavima za rad u stvarnom vremenu.
Istraženi su postojeći formalni algoritmi za raspoređivanje u preopterećenim sustavima za rad u stvarnom vremenu, kao i metode za smanjenje opterećenja u tranzijentnim i permanentnim uvjetima preopterećenja.
Implementiran je radni okvir za evoluciju heuristika za raspoređivanje koristeći genetsko programiranje.
U implementirani radni okvir integrirana je potpora za višekriterijsku optimizaciju (NSGA-II algoritam) i kooperativnu koevoluciju.
Rezultati evoluiranih heuristika analizirani su pomoću simulatora izvođenja periodičkih zadataka koji koristi raspoređivanje s istiskivanjem.
Sučelje za generiranje setova zadataka za učenje i testiranje implementirano je koristeći UUniFast algoritam.
Rezultati testiranja predstavljene tehnike uspoređeni su s rezultatima algoritma \textit{Red Tasks as Late as Possible} (RLP) s obzirom na postignutu kvalitetu usluge.
Primjenom evoluirane heuristike na veliki broj sintetski generiranih setova zadataka,
kvaliteta usluge u uvjetima preopterećenja povećana je do 50\% u odnosu na RLP algoritam.
Rezultati implementiranog radnog okvira verificirani su usporedbom s rezultatima postojećeg radnog okvira za evolucijsko računanje.
Napravljena je modifikacija operacijskog sustava FreeRTOS za korištenje evoluiranih heuristika za raspoređivanje zadataka.

\kljucnerijeci{genetsko programiranje, raspoređivanje zadataka, sustavi za rad u stvarnom vremenu.}
\end{sazetak}

\end{document}


 \begin{thebibliography}{1}

  \bibitem{cb1998} Caccamo, M. Buttazzo, G. {\em Optimal Scheduling for Fault-Tolerant and
Firm Real-Time Systems}  1998.

  \bibitem{impj}  The Japan Reader {\em Imperial Japan 1800-1945} 1973:
  Random House, N.Y.

  \bibitem{norman} E. H. Norman {\em Japan's emergence as a modern
  state} 1940: International Secretariat, Institute of Pacific
  Relations.

  \bibitem{fo} Bob Tadashi Wakabayashi {\em Anti-Foreignism and Western
  Learning in Early-Modern Japan} 1986: Harvard University Press.

  \end{thebibliography}