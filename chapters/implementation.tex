\chapter{Implementation Overview}
\section{Generating Test Task Sets}
For testing the performance of the considered algorithms and heuristics, it is needed to generate a large amount of syntetic test tasks.
The simulation results should not be biased by the task generation method, so selecting the approach to random generation of the task set requires special attention.
An important factor in generating syntetic task sets is the probability density function of the random variables used to generate the task set parameters \cite{bini2005measuring}.

The first step to generating the task set parameters is generating the task periods.
Treating the task periods as random variables does not reflect the characteristics of real applications, since task periods are defined by the user and enforced by the operating system \cite{bini2005measuring}.
However, for testing the scheduling heuristics without any a priori knowledge about their future applications and the characteristics of the corresponding environment, asuuming task periods as random variables with a uniform distribution is acceptable.

After selecting the task periods, it is required to generate task computation times, according to a given distribution.
The computation time $c_i$ is assumed to have a uniform distribution scaled by a factor $T_i$:
\begin{equation*}
c_i \sim \mathcal{U}[0, T_i].
\end{equation*}
This is equivalent to assuming each task utilization $u_i$ to have a uniform distribution in the interval $[0, 1]$.

For testing purposes, it is often desireable to generate the task utilization values in a way that they correspond to the given total processor utilization value.
The stated feature is accomplished by generating the individual utilization $u_i$ with an uniform distribution with the interval $[0, \overline{U}]$, subject to the constraint:
\begin{equation*}
\sum_{i=1}^{N}u_i = \overline{U}.
\end{equation*}
The utilization disparity in a task set can be expressed by a parameter called \textit{U-difference}:
\begin{equation*}
\delta = \frac{max_i\{u_i\}-min_i\{u_i\}}{\sum_{i=1}^{n}u_i}.
\end{equation*}
If $\delta = 0$, all utilization factors are the same, whereas $\delta=1$ denotes the maximum degree of difference \cite{bini2005measuring}.

To efficiently generate utilization factors for the task set with given utilization factor $\overline{u}$, where $\delta \rightarrow 1$, the \textit{Uunifast} algorithm is used.

\subsection{Uunifast Algorithm}
The UUniFast algorithm is built on the consideration that the probability density function of the sum of independent random variables is given by the convolution of ther probability dension functions \cite{bini2005measuring}.
In every algorithm iteration, a value of the sum of variables is randomly generated.
The single utilization factor $u_i$ is set equal to the difference between $\overline{u}$ and the generated value.
The complexity of the algorithm is $O(n)$.
A pseudocode describing the algorithm is shown in code listing \ref{uunifast}.
The algorithm inputs are the number of variables \texttt{n} and mean utilization factor, \texttt{mean\_u}.
\begin{algorithm}
\caption{Uunifast algorithm.\label{uunifast}}
\begin{algorithmic}
\STATE sum\_u = mean\_u
\FOR{i=1 \TO n-1}
\STATE next\_sum\_u = sum\_u * rand \^{} (1/(n-i))
\STATE vec\_u[i] = sum\_u - next\_sum\_u
\STATE sum\_u = next\_sum\_u
\ENDFOR
\STATE vec\_u[n] = sum\_u
\end{algorithmic}
\end{algorithm}
Figure \ref{uunifast:fig} illustrates the values of 10000 utilization tuples generated by the UUniFast algorithm.
The value of \texttt{n} is set to 3, while the mean utilization factor $\overline{u}$ is set to 1.
The density of the generated values is uniform in the interval $[0, 1]$.
\begin{figure}[ht]
    \centering
    \includegraphics[width=1\textwidth]{images/rsz_uunifast.png}
    \caption{Utilization tuples generated by the UUniFast algorithm.}
    \label{uunifast:fig}
\end{figure}

The generation of the test sets of periodic tasks is implemented in the \texttt{UunifastCreator} class by the \texttt{create\_test\_set()} method.
The required number of tasks and the overload factor are set through the class constructor.
First, the task utilizations are created by the \texttt{generate\_utils()} method which implements the algorithm described in code listing \ref{uunifast}.
The task periods are generated as random variables with logarithmic uniform distribution.
This is implemented in the \texttt{generate\_log\_uniform()} method.

\section{Heuristics Evolution}
The genetic programming evolution is implemented in the \texttt{GeneticAlgorithm} class.
The population size and generation number parameters are set through the class constructor.
Each individual is represented as an object of class \texttt{TreeSolution}.
This class contains two variable members: the fitness value of the individual and a pointer to the tree genotype.

\subsection{Genotype Description}
The priority functions evolved by genetic programming are represented in the form of a tree.
The tree genotypes correspond to arithmetic expressions used to calculate the priority of each job.
As described in the second chapter, a tree genotype consists of function and terminal nodes.
The terminal nodes used for assembling a priority function are listed in the following table.
\begin{table}[]
\centering
\begin{tabular}{|
>{\columncolor[HTML]{EFEFEF}}c |
>{\columncolor[HTML]{FFFFFF}}l |}
\hline
\textbf{Terminal name} & \multicolumn{1}{c|}{\cellcolor[HTML]{EFEFEF}\textbf{Definition}} \\ \hline
pt                     & nominal processing time of a job ($p\_j$)                          \\ \hline
dd                     & due date ($d\_j$)                                                  \\ \hline
w                      & weight ($w\_j$)                                                    \\ \hline
SL                     & positive slack, $SL = max\{d\_j - p\_j - time, 0\}$                \\ \hline
Nr                     & number of remaining (unscheduled) jobs                           \\ \hline
SPr                    & sum of processing times of all remaining jobs                    \\ \hline
SD                     & sum of due dates of all jobs                                     \\ \hline
\end{tabular}
\end{table}
The values of the variables in terminal node set depend not only on the parameters of the current job that is being dispatched, but also on the parameters of remaining (unscheduled) jobs.

The following table contains a list of the function nodes.
\begin{table}[H]
\centering
\label{tbl:functions}
\begin{tabular}{|
>{\columncolor[HTML]{EFEFEF}}c |
>{\columncolor[HTML]{FFFFFF}}c |}
\hline
\textbf{Function name} & \cellcolor[HTML]{EFEFEF}\textbf{Definition}                  \\ \hline
ADD, SUB, MUL, DIV     & Addition, subtraction, multiplication and protected division \\ \hline
POS                    & $POS(a) = max\{a, 0\}$                                         \\ \hline
\end{tabular}
\end{table}
Protected division node is a modified division operator which checks whether the second argument (denominator) is zero before performing division.
If the second argument is zero, it returns the value 1, regardless of the value of the first argument.

The genotype primitives are described by objects of the base class \texttt{AbstractNode}.
Child nodes of each primitive are stored as a vector of pointers to \texttt{AbstractNode} objects.
% The \texttt{children} vector is a member of the \texttt{AbstractNode} class.
Parameters and actions for specific primitives are handled by the derived classes, where every class corresponds to a function or terminal node.
Each class contains an \texttt{execute()} method which recursively invokes the same method upon every child node.
In terminal nodes, the \texttt{execute()} method returns the value of the corresponding task parameter.

Generating the tree genotypes is handled by the methods of the \texttt{TreeConstructor} class.
There are two basic approaches for generating a genotype: the \textit{full} method and the \textit{grow} method \cite{koza1992genetic}.
The \textit{full} method is used for generating full trees. 
It sets the root node as a random function or terminal node.
If root is a function node, the root's children nodes are set to random function nodes.
The procedure is the same for every node until the required depth is reached.
The nodes at final depth are set to random terminal nodes.
On the other hand, the \textit{grow} method selects any node (function or terminal) in each step.
Both approaches are implemented as methods of the \texttt{TreeConstructor} class and their prototypes are stated in code listing \ref{treeconstr}.
\begin{lstlisting}[frame=none, label={treeconstr}, caption={Functions for implementing the \textit{full} and \textit{grow} methods for creating a genotype.}, captionpos=b]
void construct_tree_full( int max_depth, AbstractNode *&root );
void construct_tree_grow( int max_depth, AbstractNode *&root );
\end{lstlisting}
The \texttt{max\_depth} argument determines the maximum depth of the tree.
Actual depth has a random value in the interval $[1, \texttt{max\_depth}]$.
Figure \ref{fullgrow} depicts the difference between the trees generated by \textit{full} and \textit{grow} methods.
\begin{figure}[ht]
    \centering
    \includegraphics[width=1\textwidth]{images/fullgrow.pdf}
    \caption{Examples of genotypes generated by the \textit{full} (left) and \textit{grow} method (right).}
    \label{fullgrow}
\end{figure}

\subsection{Operators}
The selection, crossover and mutation operators are implemented in the \texttt{SelectionOperator}, \texttt{CrossoverOperator} and \texttt{MutationOperator} classes, respectively.
Selection is performed by the \texttt{get\_members} method whose prototype is stated in code listing \ref{selection}.
\begin{lstlisting}[frame=none, label={selection}, caption={Prototype of the \texttt{get\_members} method which performs selection.}, captionpos=b]
template <typename T>
void TreeSelection<T>::get_members( std::vector<T> &population, 
	std::vector<T> &members );
\end{lstlisting}
The selected individuals are stored in the \texttt{members} vector.

Crossover operator is implemented by the \texttt{get\_children} method whose prototype is stated in code listing \ref{crossover}.
\begin{lstlisting}[frame=none, label={crossover}, caption={Prototype of the \texttt{get\_children} method which performs crossover.}, captionpos=b]
template <typename T>
void TreeCrossover<T>::get_children( std::vector<T> &parents, 
	std::vector<T> &children );
\end{lstlisting}
Crossover is performed by selecting a random subtree of every parent genotype.
This is done by invoking the \texttt{pick\_random} method implemented in \texttt{AbstractNode} class which returns a pointer to a random child node at a random depth.
The child genotypes are generated by swapping the randomly selected subtrees.
If this operation would cause the tree depth to exceed the maximum tree depth, crossover is not performed.

Mutation of a genotype is done through the \texttt{mutate\_solution} method.
Prototype of this method is stated in code listing \ref{mutation}.
\begin{lstlisting}[frame=none, label={mutation}, caption={Prototype of the \texttt{mutate\_solution} method.}, captionpos=b]
template <typename T>
void TreeMutation<T>::mutate_solution ( T &solution );
\end{lstlisting}
The first step of the \texttt{mutate\_solution} method is generating a new subtree which shall replace a randomly selected subtree of the individual.
The new subtree is generated by invoking either the \texttt{construct\_tree\_full} or the \texttt{construct\_tree\_grow} method with equal probability.
The subtree replacement is performed by the \texttt{replace\_random} method implemented in the \texttt{AbstractNode} class.
This method takes a pointer to a newly created subtree and replaces a random child at a random depth with the subtree.

\section{Heuristics Evaluation}
\subsection{Periodic Tasks Simulator}
In this work, the tasks are represented as objects of class \texttt{Task}.
The following parameters are set through the constructor:
\begin{itemize}
	\item task period,
	\item duration,
	\item relative due date,
	\item weight,
	\item skip factor,
	\item task ID.
\end{itemize}
The task priority computed by the heuristic is stored in the \texttt{priority} member variable.
Additional parameters used for tracking and managing task execution include:
\begin{itemize}
	\item instance counter,
	\item arrival time,
	\item absolute due date,
	\item state,
	\item tardiness.
\end{itemize}
When a new task is created, these parameters are initialized by the \texttt{initialize\_task()} method.
The \texttt{instance\_counter} variable is used for tracking which instance of a task is active and computing the arrival time and due date of the next instance.
Upon the appearance of a new task, its absolute due date and state parameters are updated.
The state parameter is used in skip-over task model and it is represented by a variable of enumerated type. Its initial value is set to \texttt{RED}.
The states of the following instances are set by the \texttt{update\_rb\_params()} method.
If the skipover model is not being used, the task tardiness is tracked.
The tardiness value of a task is updated upon the end of the task execution through the \texttt{update\_tardiness()} method.

\section{Support for Multicriterial Optimization and Cooperative Coevolution}
\section{Integrating Evolved Scheduling Heuristics into FreeRTOS}
\subsection{FreeRTOS Task Management}
FreeRTOS is an open-source real-time operating system designed for embedded systems. 
Its main features include portability, simplicity of the source code, and binary code compactness. 

The minimum FreeRTOS kernel code is contained in three source files \cite{brown2012architecture}. 
The code that handles task creating and scheduling is situated in the source file \verb$tasks.c$ and header file \verb$task.h$.

The tasks are managed through the Task Control Block (TCB) structure. 
A TCB corresponding to each task contains all information necessary to completely describe the task state. 
The TCB fields include task name, initial priority, unique TCB number and a pointer to the top of the task's stack. 
When a task is added to a list, it is represented by a pointer to a \verb$ListItem$ object. 
The TCB structure contains two \verb$ListItem$ objects: \verb$xStateListItem$ and \verb$xGenericListItem$.

A task in FreeRTOS can exist in one of the following states: deleted, suspended, ready, blocked and running. 
Figure \ref{freertos:state} shows a state diagram for FreeRTOS tasks. 

\begin{figure}[ht]
    \centering
    \includegraphics[width=0.70\textwidth]{images/freertos_fsm.pdf}
    \caption{State diagram of FreeRTOS tasks.}
    \label{freertos:state}
\end{figure}

Task states are tracked implicitly by placing tasks in the appropriate lists: ready list, suspended list, etc. As a task changes state, it is simply moved from one list to another 
\cite{brown2012architecture}.

A task is created by the \verb$xTaskCreate()$ function. 
The user-defined parameters required to create a task include: 
\begin{itemize}
	\item a pointer to the function that implements the task,
	\item the task name,
	\item the depth of the task's stack,
	\item the task's priority,
	\item a pointer to any parameters needed by the task function.
\end{itemize}
The \verb$xTakCreate()$ first allocates memory for the task's TCB and stack.
Next, the TCB fields are initialized with the task name, priority and stack depth from function parameters. 
Finally, a pointer to the top of the task's stack is initialized and the stack is populated with a \textit{dummy frame}. 
In that way, the task is prepared for its first context switch \cite{goyette2007analysis}.

After all the required tasks have been created, the FreeRTOS scheduler is started by a call to the 
\verb$vTaskStartScheduler()$ function. 
First, the Idle task with the lowest priority is created.
The global timer \verb$xTickCount$ is set to zero. 
The \verb$vTaskStartScheduler()$ function then passes control to the \verb$xTaskStartScheduler()$ in the Hardware Abstraction Layer (HAL), which configures the timer interrupt needed for invoking the scheduler. 
The HAL scheduler is also in charge of restoring the context of the currently selected task
\cite{goyette2007analysis}. 

Newly created tasks are placed into the ready state and added to the ready list. 
The ready list is implemented as an array of task lists:
\begin{lstlisting}[frame=none, label={lst:readylist}, caption={Ready task list}, captionpos=b]
static List_t pxReadyTasksLists[configMAX_PRIORITIES];
\end{lstlisting}
The elements of the \verb$pxReadyTasksLists$ array are lists of tasks that have the same priority, from \verb$0$ to \verb$configMAX_PRIORITIES-1$.
An example of a ready list is shown in figure \ref{freertos:ready}.
There are three priority levels in the list: task A has priority 0, no tasks have priority 1 and tasks B, C and D have priority 2. 

\begin{figure}[ht]
    \centering
    \includegraphics[width=0.70\textwidth]{images/ready_list.pdf}
    \caption{A schematic view of the FreeRTOS Ready List. Modified from \cite{brown2012architecture}.}
    \label{freertos:ready}
\end{figure}
When a new task is added to the ready list, its \verb$xStateListItem$ is inserted at the end of the 
associated priority level list.

A task in the running state is identified by the \verb$pxCurrentTCB$ variable, which is updated at every system tick interrupt. 
Every time the tick interrupt occurs, the \verb$xTaskSwitchContext()$ function is called and it selects the highest-priority ready task.
After the highest priority level is determined,
the highest-priority task is selected by the \verb$listGET_OWNER_OF_NEXT_ENTRY()$ function.
The function traverses the priority level's ready list and assigns the next ready task to the \verb$pxCurrentTCB$ variable.

The tasks enter the blocked state when they are waiting for time related or synchronization events. 
A task can be placed into the blocked state by calling the \verb$vTaskDelay()$ and \verb$vTaskDelayUntil()$ API functions. 
The \verb$vTaskDelayUntil()$ function defines the frequency at which the task is periodically executed and therefore it can be used to implement periodic tasks \cite{carraro2016implementation}.
The function sets the value of the tick at which the task will activate into the \verb$xStateListItem$ element and places it in the delayed list:
\begin{lstlisting}[frame=none, label={lst:delay}, caption={Transition to blocked state.}, captionpos=b]
listSET_LIST_ITEM_VALUE(&(pxCurrentTCB->xStateListItem), 
						xTimeToWake);
vListInsert(pxDelayedTaskList, &(pxCurrentTCB->xStateListItem));
\end{lstlisting}
The \verb$vListInsert()$ function sorts the elements by the \verb$xStateListItem$ values. 
At every increment of the tick count, it must be checked whether a task needs to be unblocked. 
This is implemented in the \verb$xTaskIncrementTick()$ function which is called from the HAL every time the timer interrupt occurs.
The nearest unblock tick value is stored in the \verb$xNextUnblockTime$ variable, and it corresponds to the unblock time of the first element of the delayed list.
When the current tick value achieves the \verb$xNextUnblockTime$ value, the first task from the delayed list is retrieved by the \verb$listGET_OWNER_OF_HEAD_ENTRY()$ function.
The task is placed in the ready list and the function return value signifies that a context switch is needed.

The suspended state is assumed when the \verb$vTaskSuspend()$ API function is called and the tasks are switched back from suspended state by \verb$vTaskResume()$ function.

\subsection{FreeRTOS Scheduler Modification}
FreeRTOS uses a static priority policy for task scheduling. 
To achieve dynamic priority assignment, the FreeRTOS task management subsystem must be modified. 
This is possible by modifying the existing FreeRTOS functions and data structures, but also adding new objects that allow dynamic task priorities to be managed. 

The general idea of implementing a dynamic scheduler is to create a new ready list which contains tasks ordered by a custom parameter defined in \verb$xTaskStateItem$ object.
In this case, \verb$xTaskStateItem$ shall contain the task priority computed by the evaluated heuristic and the list will be sorted in increasing priority value. 
The priority of the Idle task is set to some arbitrary value, significantly greater than the other tasks' priorities. 

To determine whether dynamic scheduling based on the evolved heuristic is used, a configuration variable \verb$configUSE_GP_SCHEDULER$ is added to the \verb$FreeRTOSConfig.h$ file. 

As described in the previous sections, the task parameters used for priority computation include the task period, duration, deadline and weight. 
These parameters need to be added to the TCB structure. 
Additionally, the task priority value computed by the heuristic is added, as well as the \verb$xRemainingTicks$ variable used for tracing task execution.
\begin{lstlisting}[frame=none, label={TCB}, caption={Modification of the TCB.}, captionpos=b]
typedef struct tskTaskControlBlock
{
	...
	#if( configUSE_GP_SCHEDULER == 1 )
		TickType_t xTaskPeriod;
		TickType_t xTaskDuration;
		TickType_t xDeadline;
		TickType_t xRemainingTicks;
		double xTaskWeight;
		double xPriorityValue;
	#endif
} tskTCB;
\end{lstlisting}
The additional parameters are set through the \verb$xTaskPeriodicCreate()$ function, which is a modified version of the standard \verb$xTaskCreate()$.
\begin{lstlisting}[frame=none, label={periodicCreate}, caption={The \texttt{xTaskPeriodicCreate()} function prototype.}, captionpos=b]
BaseType_t xTaskPeriodicCreate(	TaskFunction_t pxTaskCode,
						const char * const pcName,		
						const configSTACK_DEPTH_TYPE usStackDepth,
						void * const pvParameters,
						UBaseType_t uxPriority,
						TaskHandle_t * const pxCreatedTask,
						TickType_t period,
						TickType_t duration,
						uint32_t weight ) PRIVILEGED_FUNCTION
\end{lstlisting}
The user-defined parameters are added to the new task's TCB structure. 
\begin{lstlisting}[frame=none, label={TCB_params}, caption={Adding the user-defined task parameters to the TCB structure.}, captionpos=b]
pxNewTCB->xTaskPeriod = period;
pxNewTCB->xTaskDuration = duration;
pxNewTCB->uTaskWeight = weight;
pxNewTCB->xDeadline= period;
pxNewTCB->xRemainingTicks = duration;
\end{lstlisting}
After the task priority value is computed, this value is assigned to the task's \\\verb$xStateListItem$ element. 
\begin{lstlisting}[frame=none, label={init_priority}, caption={Assigning the task priority to the \texttt{xStateListItem} element.}, captionpos=b]
vTaskComputePriority( pxNewTCB );
listSET_LIST_ITEM_VALUE( &((pxNewTCB)->xStateListItem), 
						(pxNewTCB)->xPriorityValue );
\end{lstlisting}
Next, the new ready tasks list must be declared.
\begin{lstlisting}[frame=none, label={ready_list}, caption={Declaration of the new ready tasks list.}, captionpos=b]
#if( configUSE_GP_SCHEDULER == 1 )
	PRIVILEGED_DATA static List_t xReadyTasksListGP;
#endif /* configUSE_GP_SCHEDULER */
\end{lstlisting}
Before adding a task to the ready list, its priority value is computed. 
Since the Idle task is added to the ready list in the same way as other tasks, it has to be distinguished in some manner. 
While creating the Idle task, its duration is set to 0.
Therefore, before assigning a priority value to a task, its duration value is checked. 
An example of the priority computation according to an arbitrary heuristic is given in the following code listing. 
In this example, priority is computed according to the expression:
\begin{align*}
p_i = d_i + c_i + max\{ d_i - c_i - current\_time, 0 \}.
\end{align*}
The Idle task priority is set to 1000.
\begin{lstlisting}[frame=none, label={ready_list}, caption={Macro function for priority computation.}, captionpos=b]
#define vTaskComputePriority( pxTCB )
{
	pxTCB->xPriorityValue = ( pxTCB->xTaskDuration != 0 ) ? 
		( pxTCB->xDeadline + pxTCB->xTaskDuration + 
		max( pxTCB->xDeadline - pxTCB->xTaskDuration, 0 )) : 1000;
}	
\end{lstlisting}
The \verb$prvAddTaskToReadyList()$ function is modified in a way that the \verb$vListInsert()$ method is used instead of \verb$vListInsertEnd()$. 
The elements are sorted by the \\\verb$xStateListItem$ element, so the lowest priority tasks are placed at the beginning of the list. 
\begin{lstlisting}[frame=none, label={ready_add}, caption={Adding a new task to the ready list.}, captionpos=b]
#if ( configUSE_GP_SCHEDULER == 1 )
	#define prvAddTaskToReadyList( pxTCB )
	{
		vTaskComputePriority( pxTCB );
		listSET_LIST_ITEM_VALUE( &((pxTCB)->xStateListItem), 
								(pxTCB)->xPriorityValue );
		vListInsert(&(xReadyTasksListGP), 
					&((pxTCB)->xStateListItem) );
	}																								
#else 																								
	#define prvAddTaskToReadyList( pxTCB )
	{
		vListInsertEnd(&(pxReadyTasksLists[(pxTCB)->uxPriority] ), 
						&((pxTCB)->xStateListItem));
	}
#endif
\end{lstlisting}
The \verb$prvInitialiseTaskLists()$ must be called in order to initialise the elements of the ready tasks list.
\begin{lstlisting}[frame=none, label={ready_list}, caption={Declaration of the new ready tasks list.}, captionpos=b]
#if( configUSE_GP_SCHEDULER == 1 )
	vListInitialise( &(xReadyTasksListGP) );
#endif /* configUSE_GP_SCHEDULER */
\end{lstlisting}
When a new task is added to the ready list while the scheduler is not running, its priority value is compared to the priority of the running task. 
If the new task has a higher priority, it is set as the currently running task.
Therefore, a modification in the \verb$prvAddNewTaskToReadyList()$ function is required in order to include the comparison between computed priority values. 
\begin{lstlisting}[frame=none, label={newtask}, caption={Modification of the \texttt{prvAddNewTaskToReadyList()} function.}, captionpos=b]
#if( configUSE_GP_SCHEDULER == 1 )
	vListInitialise( &(xReadyTasksListGP) );
#endif /* configUSE_GP_SCHEDULER */
\end{lstlisting}
The \verb$vTaskStartScheduler()$ is also modified in order to manage the creation of the Idle task.
\begin{lstlisting}[frame=none, label={idle}, caption={Creation of the Idle task.}, captionpos=b]
#if( configUSE_GP_SCHEDULER == 1 )
{
	xReturn = xTaskPeriodicCreate( prvIdleTask,
					configIDLE_TASK_NAME,
					configMINIMAL_STACK_SIZE,
					( void* ) NULL,
					( tskIDLE_PRIORITY | portPRIVILEGE_BIT ),
					&xIdleTaskHandle,
					0,
					0,
					0 );
}
#else 
	...
\end{lstlisting}
The \verb$vTaskSwitchContext()$ function is modified in a way that it selects the first element of the ready list as the currently running task, instead of calling the \\\verb$taskSELECT_HIGHEST_PRIORITY_TASK()$ function.
\begin{lstlisting}[frame=none, label={switchcontext}, caption={\texttt{vTaskSwitchContext()} modification.}, captionpos=b]
#if( configUSE_GP_SCHEDULER == 1 )
{
	pxCurrentTCB =
		(TCB_t *)listGET_OWNER_OF_HEAD_ENTRY(&(xReadyTasksListGP));
}
#else 
{
	taskSELECT_HIGHEST_PRIORITY_TASK();
}
#endif
\end{lstlisting} 
Since absolute deadline value is required for priority computation, this parameter must be updated at each end of a task instance.
If the \verb$xRemainingTicks$ variable reached zero before the task is switched out, \verb$xDeadline$ parameter is updated.
\begin{lstlisting}[frame=none, label={taskfinish}, caption={\texttt{Updating the \texttt{xDeadline} parameter at the end of task instance}.}, captionpos=b]
#if( configUSE_GP_SCHEDULER == 1 )
{
	if( pxCurrentTCB->xRemainingTicks == 0 ) 
	{
		pxCurrentTCB->xRemainingTicks = 
			pxCurrentTCB->xTaskDuration;
		pxCurrentTCB->xDueDate += pxCurrentTCB->xTaskPeriod;
	}
}
#endif
\end{lstlisting}
Finally, the last modifications are applied to the \verb$xTaskIncrementTick()$ function. 
The previous subsection described how a task is unblocked and placed into the ready list.
When a task is blocked, its \verb$xStateListItem$ element refers to the unblock tick value. 
Therefore, before placing an unblocked task into the ready list, its \verb$xStateListItem$ element must be assigned to the priority value.
Priority of the unblocked task is compared to the running task's priority in order to determine whether a context switch is needed.
Hence, the priority of the task in running state is also updated.
\begin{lstlisting}[frame=none, label={switchcontext}, caption={Managing an unblocked task.}, captionpos=b]
#if( configUSE_GP_SCHEDULER == 1 )
{
	vTaskComputePriority( pxTCB );
	listSET_LIST_ITEM_VALUE( &((pxTCB)->xStateListItem), 
							pxTCB->xPriorityValue );
	vaskComputePriority( pxCurrentTCB );
}
#endif
\end{lstlisting}
