\chapter{Introduction}
Real-time systems have a wide range of application, including monitoring systems in industrial plants, automotive, space missions, multimedia systems and consumer electronics.
Reliability of these systems depends on the timeliness and predictability of task execution.
The scope of this thesis considers firm real-time systems, where the completion of a tardy task brings no value to the system.
A typical example of such system is a multimedia system.
Although a certain degree of deadline miss ratio is allowed, the system will eventually suffer from a failure if consecutive instances of a task fail to complete before their deadlines.

There is a handful of formal algorithms for scheduling periodic firm real-time tasks proposed in literature.
The general idea of these algorithms is to resolve system overload by skipping some task instances in order to achieve the feasible load.
Making optimal use of skips has been proven to be an NP-hard problem \cite{queudet2012quality}.

The idea of this thesis is to investigate the possibility of using machine learning to optimize a heuristic for scheduling skippable periodic tasks.
More specifically, the goal is to build a scheduling heuristic such that the number of successfully completed task instances between skipped instances is maximized.
The scheduling algorithm is structured in two components: a meta-algorithm and a priority function.
A meta-algorithm is a method of assigning task instances (jobs) to resources with respect to task properties and system constraints.
It is defined for a given scheduling environment.
In this thesis, the considered environment is a single-machine environment where $n$ periodic tasks are processed on a single resource. 
The meta-algorithm encapsulates the priority function which defines priority values for the tasks.
This thesis deals with generating the priority function by using genetic programming.

The thesis is organized as follows.
Chapter \ref{genprog} gives an introduction to optimization techniques, evolutionary algorithms and genetic programming.
The main concepts of multi-objective optimization and cooperative coevolution are described.
The formal methods for handling transient and permanent overload conditions in real-time tasks are described in Chapter \ref{fm}.
The main focus is set on skippable periodic tasks.
The skip-over task model is described, as well as the scheduling algorithms based on this model.
Chapter \ref{implementation} presents in detail the implementation of a framework for genetic programming in C++ language.
The interfaces for applying evolved heuristic to task scheduling problem and evaluating its performace are also described.
Chapter \ref{results} gives a comparison of the implemented solution to formal algorithms and an existing framework for evolutionary computation.
Additionally, the results of integrating the evolved priority function to FreeRTOS scheduler are presented.