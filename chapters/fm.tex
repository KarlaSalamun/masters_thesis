\chapter{Formal Methods of Scheduling Overloaded Real-Time Systems}
%%%% TODO ovdje pisati o podjeli zadataka na firm, soft, hard???

\section{Transient Overload}
\subsection{Aperiodic Overloads}
\subsection{Task Overruns}

%%%%%%%%%%%%%%%%%%%%%%%%%%%%%%%%%%%%%%%%%%%%%%%%%%%%%%%%%%%%%%%%%%%%%%%%%%%%%%%%%%%%%%%%%%%%%%
\section{Permanent Overload}

Permanent overload in periodic task systems occurs when the total utilization factor of the periodic task set is greater than one \cite{buttazzo2011hard}. 
This condition occurs either because of wrong estimation of the task execution time, unexpected activation of new periodic tasks, or the increase of activation rate of current tasks.
Methods for reducing permanent overload are the following:
\begin{itemize}
	\item{job skipping,}
	\item{period adaptation,}
	\item{service adaptation.}
\end{itemize}

\subsection{Job Skipping}
Job skipping method reduces system load by skipping some jobs in the task set. Jobs to be skipped are assigned in a way that the remainig jobs can be scheduled within their deadlines.
This method is applicable for firm tasks, as they allow a certain miss ratio. 
The quantified ratio of tasks that may not be executed is directly related to Quality of Service 
(QoS) metric. 
The algorithms presented in this section address the task miss ratio as a QoS concern, with the purpose of maximizing the QoS of periodic tasks.

A task model suitable for job skipping method is known in literature as the firm periodic model.
It was first described by Koren and Shasha \cite{koren1995skip}.
According to this model, every task is described as following:
\begin{align*}
\tau_i(C_i, T_i, D_i, S_i)
\end{align*}
where $C_i$ is the worst-case computation time, $T_i$ is the task period, $D_i$ is the relative deadline (assumed equal to the period). $S_i$ is the skip parameter and it marks the minimum distance between two consecutive skips, \(2 \leq S_i \leq \infty\).

An example of schedule with task skipping is given in figure \ref{skipover}. 
Since the utilization factor equals 
\begin{equation*}
U_p = \sum_{i=1}^{n} \frac{C_i}{T_i} = 1.17,
\end{equation*}
the system is permanently overloaded. 
However, if task \(\tau_2\) is skipped every three instances, the overload is resolved.
\\
\begin{figure}[ht]
    \centering
    \includegraphics[width=0.9\textwidth]{images/skipover.pdf}
    \caption{An example of resolving permanent overload by skipping method.}
    \label{skipover}
\end{figure}

Corresponding to the stated task model, every job of a periodic task can be red or blue.
A red job must be completed within its deadline, whereas a blue job can be aborted at any time 
\cite{buttazzo2011hard}. 
With respect to the skip parameter $S_i$, each scheduling algorithm needs to fulfill two conditions:
\begin{itemize}
	\item if a blue job is skipped, then the next \(S_i - 1\) jobs must be red,
	\item if a blue job completes successfully, the next job is also blue.
\end{itemize}
The authors proved that the problem of determining whether a set of periodic skippable tasks is schedulable is NP-hard \cite{koren1995skip}.

% Although the algorithms are not optimal, they become optimal under a certain condition, called the 
% \textit{deeply-red} condition. 
% The \textit{deeply-red} condition is the worst case condition for periodic task sets:
% all tasks are synchronously activated and the first \(S_i - 1\) instances of every task $\tau_i$ are red. 

For the given set of skippable periodic tasks \( \Gamma = {\tau_i(T_i,C_i,S_i)} \), the necessary schedulability condition is the following:
\begin{equation}
\sum_{i=1}^{n} \frac{C_i(S_i-1)}{T_i S_i} \leq 1.
\end{equation}

A sufficient condition for guaranteeing schedulability of a set of skippable tasks can be stated 
using the equivalent utilization factor, which is formally described in the following definition:
\newtheorem{mydef}{Definition}
\begin{mydef}
Given a set \( \Gamma = {\tau_i(T_i,C_i,S_i)} \) of n skippable periodic tasks, the equivalent utilization factor is defined as:
\begin{equation*}
U_p^\ast = \max_{L \geq 0} \frac{\sum_{i}D(i, [0,L])}{L}
\end{equation*}
where
\begin{equation*}
D(i, [0,L]) = \left\lfloor \frac{L}{T_i} - \frac{L}{T_i \cdot S_i}\right\rfloor \cdot C_i.
\end{equation*}
\end{mydef}

According to the sufficient schedulability condition, a set of skippable periodic tasks is schedulable if \( U_p^\ast \leq 1 \).

\subsubsection{RTO Algorithm}
\subsubsection{BWP Algorithm}
\subsubsection{RLP Algorithm}


\subsection{Period Adaptation}
\subsection{Service Adaptation}