\chapter*{Conclusion}
This thesis presented a method for optimizing the skip factor of skippable periodic tasks.
The considered task model is typical for multimedia applications, where a dynamic scheduler is required.
Furthermore, the scheduler must be able to handle overload conditions and provide frequent schedule modifications.
Formal algorithms for scheduling skippable tasks (i.e. RLP and RLP/T) bring a significant overhead when applied to large amount of tasks, due to their computational complexity \cite{onqos}.
For that reason, heuristic scheduling is more acceptable in real-time systems with timeliness constraints due to its simplicity and ability to build a schedule with no overhead.
Genetic programming is suitable for evolving a heuristic which represents a priority function used for building a schedule.

Schedules built by evolved heuristics were compared to the RLP schedules with respect to QoS metric.
In overload conditions, the results of the evolved heuristic surpass the results achieved by RLP algorithm by up to 50\%.

However, heuristic scheduling does not guarantee a minimum number of completed task instances between skipped ones.
In other words, skipping consecutive task instances is allowed, which can totally degrade system performance.
Therefore, using evolved heuristics for scheduling is applicable only in situations where task set parameters are known beforehand.

\newcommand{\namesigdate}[2][5cm]{%
  \begin{tabular}{@{}p{#1}@{}}
    #2 \\[2\normalbaselineskip] \hrule \\[15pt]
  \end{tabular}}

\vspace*{\fill} \noindent \hfill \namesigdate{Karla Salamun}